\ifx\allfiles\undefined
\documentclass[12pt, a4paper, oneside, UTF8]{ctexbook}
\def\path{../../config}
\usepackage{amsmath}
\usepackage{amsthm}
\usepackage{amssymb}
\usepackage{array}
\usepackage{xcolor}
\usepackage{graphicx}
\usepackage{mathrsfs}
\usepackage{enumitem}
\usepackage{geometry}
\usepackage[colorlinks, linkcolor=black]{hyperref}
\usepackage{stackengine}
\usepackage{yhmath}
\usepackage{extarrows}
\usepackage{tikz}
\usepackage{pgfplots}
\usetikzlibrary{decorations.pathreplacing, positioning}
\usepgfplotslibrary{fillbetween}
% \usepackage{unicode-math}
\usepackage{esint}
\usepackage[most]{tcolorbox}

\usepackage{fancyhdr}
\usepackage[dvipsnames, svgnames]{xcolor}
\usepackage{listings}

\definecolor{mygreen}{rgb}{0,0.6,0}
\definecolor{mygray}{rgb}{0.5,0.5,0.5}
\definecolor{mymauve}{rgb}{0.58,0,0.82}
\definecolor{NavyBlue}{RGB}{0,0,128}
\definecolor{Rhodamine}{RGB}{255,0,255}
\definecolor{PineGreen}{RGB}{0,128,0}

\graphicspath{ {figures/},{../figures/}, {config/}, {../config/} }

\linespread{1.6}

\geometry{
    top=25.4mm, 
    bottom=25.4mm, 
    left=20mm, 
    right=20mm, 
    headheight=2.17cm, 
    headsep=4mm, 
    footskip=12mm
}

\setenumerate[1]{itemsep=5pt,partopsep=0pt,parsep=\parskip,topsep=5pt}
\setitemize[1]{itemsep=5pt,partopsep=0pt,parsep=\parskip,topsep=5pt}
\setdescription{itemsep=5pt,partopsep=0pt,parsep=\parskip,topsep=5pt}

\lstset{
    language=Mathematica,
    basicstyle=\tt,
    breaklines=true,
    keywordstyle=\bfseries\color{NavyBlue}, 
    emphstyle=\bfseries\color{Rhodamine},
    commentstyle=\itshape\color{black!50!white}, 
    stringstyle=\bfseries\color{PineGreen!90!black},
    columns=flexible,
    numbers=left,
    numberstyle=\footnotesize,
    frame=tb,
    breakatwhitespace=false,
} 

\lstset{
    language=TeX, % 设置语言为 TeX
    basicstyle=\ttfamily, % 使用等宽字体
    breaklines=true, % 自动换行
    keywordstyle=\bfseries\color{NavyBlue}, % 关键字样式
    emphstyle=\bfseries\color{Rhodamine}, % 强调样式
    commentstyle=\itshape\color{black!50!white}, % 注释样式
    stringstyle=\bfseries\color{PineGreen!90!black}, % 字符串样式
    columns=flexible, % 列的灵活性
    numbers=left, % 行号在左侧
    numberstyle=\footnotesize, % 行号字体大小
    frame=tb, % 顶部和底部边框
    breakatwhitespace=false % 不在空白处断行
}

% \begin{lstlisting}[language=TeX] ... \end{lstlisting}

% 定理环境设置
\usepackage[strict]{changepage} 
\usepackage{framed}

\definecolor{greenshade}{rgb}{0.90,1,0.92}
\definecolor{redshade}{rgb}{1.00,0.88,0.88}
\definecolor{brownshade}{rgb}{0.99,0.95,0.9}
\definecolor{lilacshade}{rgb}{0.95,0.93,0.98}
\definecolor{orangeshade}{rgb}{1.00,0.88,0.82}
\definecolor{lightblueshade}{rgb}{0.8,0.92,1}
\definecolor{purple}{rgb}{0.81,0.85,1}

\theoremstyle{definition}
\newtheorem{myDefn}{\indent Definition}[section]
\newtheorem{myLemma}{\indent Lemma}[section]
\newtheorem{myThm}[myLemma]{\indent Theorem}
\newtheorem{myCorollary}[myLemma]{\indent Corollary}
\newtheorem{myCriterion}[myLemma]{\indent Criterion}
\newtheorem*{myRemark}{\indent Remark}
\newtheorem{myProposition}{\indent Proposition}[section]

\newenvironment{formal}[2][]{%
	\def\FrameCommand{%
		\hspace{1pt}%
		{\color{#1}\vrule width 2pt}%
		{\color{#2}\vrule width 4pt}%
		\colorbox{#2}%
	}%
	\MakeFramed{\advance\hsize-\width\FrameRestore}%
	\noindent\hspace{-4.55pt}%
	\begin{adjustwidth}{}{7pt}\vspace{2pt}\vspace{2pt}}{%
		\vspace{2pt}\end{adjustwidth}\endMakeFramed%
}

\newenvironment{definition}{\vspace{-\baselineskip * 2 / 3}%
	\begin{formal}[Green]{greenshade}\vspace{-\baselineskip * 4 / 5}\begin{myDefn}}
	{\end{myDefn}\end{formal}\vspace{-\baselineskip * 2 / 3}}

\newenvironment{theorem}{\vspace{-\baselineskip * 2 / 3}%
	\begin{formal}[LightSkyBlue]{lightblueshade}\vspace{-\baselineskip * 4 / 5}\begin{myThm}}%
	{\end{myThm}\end{formal}\vspace{-\baselineskip * 2 / 3}}

\newenvironment{lemma}{\vspace{-\baselineskip * 2 / 3}%
	\begin{formal}[Plum]{lilacshade}\vspace{-\baselineskip * 4 / 5}\begin{myLemma}}%
	{\end{myLemma}\end{formal}\vspace{-\baselineskip * 2 / 3}}

\newenvironment{corollary}{\vspace{-\baselineskip * 2 / 3}%
	\begin{formal}[BurlyWood]{brownshade}\vspace{-\baselineskip * 4 / 5}\begin{myCorollary}}%
	{\end{myCorollary}\end{formal}\vspace{-\baselineskip * 2 / 3}}

\newenvironment{criterion}{\vspace{-\baselineskip * 2 / 3}%
	\begin{formal}[DarkOrange]{orangeshade}\vspace{-\baselineskip * 4 / 5}\begin{myCriterion}}%
	{\end{myCriterion}\end{formal}\vspace{-\baselineskip * 2 / 3}}
	

\newenvironment{remark}{\vspace{-\baselineskip * 2 / 3}%
	\begin{formal}[LightCoral]{redshade}\vspace{-\baselineskip * 4 / 5}\begin{myRemark}}%
	{\end{myRemark}\end{formal}\vspace{-\baselineskip * 2 / 3}}

\newenvironment{proposition}{\vspace{-\baselineskip * 2 / 3}%
	\begin{formal}[RoyalPurple]{purple}\vspace{-\baselineskip * 4 / 5}\begin{myProposition}}%
	{\end{myProposition}\end{formal}\vspace{-\baselineskip * 2 / 3}}


\newtheorem{example}{\indent \color{SeaGreen}{Example}}[section]
\renewcommand{\proofname}{\indent\textbf{\textcolor{TealBlue}{Proof}}}
\newenvironment{solution}{\begin{proof}[\indent\textbf{\textcolor{TealBlue}{Solution}}]}{\end{proof}}

% 自定义命令的文件

\def\d{\mathrm{d}}
\def\R{\mathbb{R}}
%\newcommand{\bs}[1]{\boldsymbol{#1}}
%\newcommand{\ora}[1]{\overrightarrow{#1}}
\newcommand{\myspace}[1]{\par\vspace{#1\baselineskip}}
\newcommand{\xrowht}[2][0]{\addstackgap[.5\dimexpr#2\relax]{\vphantom{#1}}}
\newenvironment{mycases}[1][1]{\linespread{#1} \selectfont \begin{cases}}{\end{cases}}
\newenvironment{myvmatrix}[1][1]{\linespread{#1} \selectfont \begin{vmatrix}}{\end{vmatrix}}
\newcommand{\tabincell}[2]{\begin{tabular}{@{}#1@{}}#2\end{tabular}}
\newcommand{\pll}{\kern 0.56em/\kern -0.8em /\kern 0.56em}
\newcommand{\dive}[1][F]{\mathrm{div}\;\boldsymbol{#1}}
\newcommand{\rotn}[1][A]{\mathrm{rot}\;\boldsymbol{#1}}

% 修改参数改变封面样式,0 默认原始封面、内置其他1、2、3种封面样式
\def\myIndex{0}


\ifnum\myIndex>0
    \input{\path/cover_package_\myIndex} 
\fi

\def\myTitle{姜晓千 2023年强化班笔记}
\def\myAuthor{Weary Bird}
\def\myDateCover{\today}
\def\myDateForeword{\today}
\def\myForeword{相见欢·林花谢了春红}
\def\myForewordText{
    林花谢了春红,太匆匆。
    无奈朝来寒雨晚来风。
    胭脂泪,相留醉,几时重。
    自是人生长恨水长东。
}
\def\mySubheading{数学笔记}


\begin{document}
\input{\path/cover_text_\myIndex.tex}

\newpage
\thispagestyle{empty}
\begin{center}
    \Huge\textbf{\myForeword}
\end{center}
\myForewordText
\begin{flushright}
    \begin{tabular}{c}
        \myDateForeword
    \end{tabular}
\end{flushright}

\newpage
\pagestyle{plain}
\setcounter{page}{1}
\pagenumbering{Roman}
\tableofcontents

\newpage
\pagenumbering{arabic}
% \setcounter{chapter}{-1}
\setcounter{page}{1}

\pagestyle{fancy}
\fancyfoot[C]{\thepage}
\renewcommand{\headrulewidth}{0.4pt}
\renewcommand{\footrulewidth}{0pt}








\else
\fi

\chapter{一元函数积分学}
\section{ 定积分的概念}

\begin{enumerate}[label=\arabic*.]
    \item 例1 (2007,数一、数二、数三)如图,连续函数$y=f(x)$在区间[-3,-2],[2,3]上的图形分别是直径为1的上、下半圆周,在区间[-2,0],[0,2]的图形分别是直径为2的下、上半圆周.
    设$F(x)=\int_0^x f(t) dt$,则下列结论正确的是:
    \begin{align*}
        (A) F(3)=-\frac{3}{4} F(-2)
    \end{align*}
    
    \begin{solution}
    【详解】
    \end{solution}
    
    \item 例2 (2009,数三)使不等式$\int_1^x\frac{\sin t}{t} dt>\ln x$成立的$x$的范围是
    \begin{align*}
        (A)\ (0,1)\quad(B)\left(1,\frac{\pi}{2}\right)\quad(C)\left(\frac{\pi}{2},\pi\right)\quad(D)(\pi,+\infty)
    \end{align*}
    
    \begin{solution}
    【详解】
    \end{solution}
    
    \item 例3 (2003,数二)设$I_1=\int_0^{\frac{\pi}{4}}\frac{\tan x}{x} dx, I_2=\int_0^{\frac{\pi}{4}}\frac{x}{\tan x} dx$,则
    \begin{align*}
        (A) I_1>I_2>1\quad(B) 1>I_1>I_2 \\
        (C) I_2>I_1>1\quad(D) 1>I_2>I_1
    \end{align*}
    
    \begin{solution}
    【详解】
    \end{solution}
\end{enumerate}

\section{ 不定积分的计算}

\begin{enumerate}[label=\arabic*.,start=4]
    \item 例5 (2009,数二、数三)计算不定积分$\int\frac{1}{1+\sqrt{\frac{1+x}{x}}}dx(x>0)$
    
    \begin{solution}
    【详解】
    \end{solution}
    
    \item 例6 求$\int\frac{1}{1+\sin x+\cos x} dx$
    
    \begin{solution}
    【详解】
    \end{solution}
\end{enumerate}

\section{ 定积分的计算}

\begin{enumerate}[label=\arabic*.,start=6]
    \item 例7 (2013,数一)计算$\int_0^1\frac{f(x)}{\sqrt{x}} dx$,其中$f(x)=\int_1^x\frac{\ln(t+1)}{t} dt$
    
    \begin{solution}
    【详解】
    \end{solution}
    
    \item 例8 求下列积分:
    \begin{align*}
        (1)\ \int_0^{\frac{\pi}{2}}\frac{1}{1+(\tan x)^{\sqrt{2}}} dx
    \end{align*}
    
    \begin{solution}
    【详解】
    \end{solution}
    
    \item 例9 求$\int_0^{\frac{\pi}{4}}\ln(1+\tan x) dx$
    
    \begin{solution}
    【详解】
    \end{solution}
\end{enumerate}

\section{ 反常积分的计算}

\begin{enumerate}[label=\arabic*.,start=9]
    \item 例10 (1998,数二)计算积分(题目内容缺失)
    
    \begin{solution}
    【详解】
    \end{solution}
\end{enumerate}

\section{ 反常积分敛散性的判定}

\begin{enumerate}[label=\arabic*.,start=10]
    \item 例11 (2016,数一)若反常积分$\int_0^{+\infty}\frac{1}{x^a(1+x)^b} dx$收敛,则
    \begin{align*}
        (A)\ a<1且\ b>1 \\
        (B)\ a>1且\ b>1 \\
        (C)\ a<1且\ a+b>1 \\
        (D)\ a>1且\ a+b>1
    \end{align*}
    
    \begin{solution}
    【详解】
    \end{solution}
    
    \item 例12 (2010,数一、数二)设$m,n$均为正整数,则反常积分$\int_0^1\frac{\sqrt[n]{\ln^2(1-x)}}{\sqrt[n]{x}} dx$的收敛性
    \begin{align*}
        (A)\ 仅与\ m\ 的取值有关 \\
        (B)\ 仅与\ n\ 的取值有关 \\
        (C)\ 与\ m,n\ 的取值都有关 \\
        (D)\ 与\ m,n\ 的取值都无关
    \end{align*}
    
    \begin{solution}
    【详解】
    \end{solution}
\end{enumerate}

\section{ 变限积分函数}

\begin{enumerate}[label=\arabic*.,start=12]
    \item 例13 (2013,数二)设函数$f(x)=\begin{cases}
        \sin x, & 0\leq x<\pi \\
        2, & \pi\leq x\leq 2\pi
    \end{cases}$,$F(x)=\int_0^x f(t) dt$,则
    \begin{align*}
        (A)\ x=\pi\ 是函数\ F(x)\ 的跳跃间断点 \\
        (B)\ x=\pi\ 是函数\ F(x)\ 的可去间断点 \\
        (C)\ F(x)\ 在\ x=\pi\ 处连续但不可导 \\
        (D)\ F(x)\ 在\ x=\pi\ 处可导
    \end{align*}
    
    \begin{solution}
    【详解】
    \end{solution}
    
    \item 例14 (2016,数二)已知函数$f(x)$在$[0,3\pi]$上连续,在$(0,3\pi)$内是函数的一个原函数,且$f(0)=0$.
    \begin{enumerate}[label=(\roman*)]
        \item 求$f(x)$在区间$[0,\frac{3\pi}{2}]$上的平均值;
        \item 证明$f(x)$在区间$[0,\frac{3\pi}{2}]$内存在唯一零点.
    \end{enumerate}
    
    \begin{solution}
    【详解】
    \end{solution}
\end{enumerate}

\section{ 定积分应用求面积}

\begin{enumerate}[label=\arabic*.,start=14]
    \item 例15 (2019,数一、数二、数三)求曲线$y=e^{-x}\sin x(x\geq 0)$与$x$轴之间图形的面积.
    
    \begin{solution}
    【详解】
    \end{solution}
\end{enumerate}

\section{ 定积分应用求体积}

\begin{enumerate}[label=\arabic*.,start=15]
    \item 例16 (2003,数一)过原点作曲线$y=\ln x$的切线,该切线与曲线$y=\ln x$及$x$轴围成平面图形$D$.
    \begin{enumerate}[label=(\roman*)]
        \item 求$D$的面积$A$;
        \item 求$D$绕直线$x=e$旋转一周所得旋转体的体积$V$.
    \end{enumerate}
    
    \begin{solution}
    【详解】
    \end{solution}
    
    \item 例17 (2014,数二)已知函数$f(x, y)$满足$\frac{\partial f}{\partial y}=2(y+1)$,且$f(y, y)=(y+1)^2-(2-y)\ln y$,求曲线$f(x, y)=0$所围图形绕直线$y=-1$旋转所成旋转体的体积.
    
    \begin{solution}
    【详解】
    \end{solution}
\end{enumerate}

\section{ 定积分应用求弧长}

\begin{enumerate}[label=\arabic*.,start=17]
    \item 例18 求心形线$r=a(1+\cos\theta)(a>0)$的全长.
    
    \begin{solution}
    【详解】
    \end{solution}
\end{enumerate}

\section{ 定积分应用求侧面积}

\begin{enumerate}[label=\arabic*.,start=18]
    \item 例19 (2016,数二)设$D$是由曲线$y=\sqrt{1-x^2}(0\leq x\leq 1)$与$x=\cos^3 t$围成的平面区域,求$D$绕$x$轴旋转一周所得旋转体的体积和表面积.
    
    \begin{solution}
    【详解】
    \end{solution}
\end{enumerate}

\section{一 定积分物理应用}

\begin{enumerate}[label=\arabic*.,start=19]
    \item 例20 (2020,数二)设边长为$2a$等腰直角三角形平板铅直地沉没在水中,且斜边与水面相齐,设重力加速度为$g$,水密度为$\rho$,则该平板一侧所受的水压力为
    
    \begin{solution}
    【详解】
    \end{solution}
\end{enumerate}

\section{二 证明含有积分的等式或不等式}

\begin{enumerate}[label=\arabic*.,start=20]
    \item 例21 (2000,数二)设函数$S(x)=\int_0^x|\cos t| dt$.
    \begin{enumerate}[label=(\roman*)]
        \item 当$n$为正整数,且$n\pi\leq x<(n+1)\pi$时,证明$2n\leq S(x)<2(n+1)$;
        \item 求$\lim_{x\to+\infty}\frac{S(x)}{x}$
    \end{enumerate}
    
    \begin{solution}
    【详解】
    \end{solution}
    
    \item 例22 (2014,数二、数三)设函数$f(x), g(x)$在区间$[a, b]$上连续,且$f(x)$单调增加,$0\leq g(x)\leq 1$.
    证明:
    \begin{enumerate}[label=(\roman*)]
        \item $0\leq\int_a^x g(t) dt\leq x-a, x\in[a, b]$;
        \item $\int_a^{a+\int_a^b g(t) dt} f(x) dx\leq\int_a^b f(x) g(x) dx$.
    \end{enumerate}
    
    \begin{solution}
    【详解】
    \end{solution}
\end{enumerate}

\ifx\allfiles\undefined
\end{document}
\fi