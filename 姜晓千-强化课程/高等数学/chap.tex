\ifx\allfiles\undefined
\documentclass[12pt, a4paper, oneside, UTF8]{ctexbook}
\def\path{../config}
\input{../config/_config}
\begin{document}
\input{../config/cover}
\else
\fi

\chapter{高等数学部分}

\section{函数 极限 连续}
\subsection{函数的性态}
\begin{remark}(有界性的判定)
    \item 连续函数在闭区间$[a,b]$上必然有界 

    \item 连续函数在开区间$(a,b)$上只需要判断端点处的左右极限,若$\lim_{x\to a^{+}}\neq \infty$ 且
    $\lim_{x\to b^{-}}\neq \infty$,则连续函数在该区间内有界.
\end{remark}

\begin{enumerate}[label=\arabic*.]
    \item  下列函数无界的是
    \begin{align*}
        (\text{A})&\quad f(x)=\frac{1}{x}\sin x, x\in(0,+\infty) \\
        (\text{B})&\quad f(x)=x\sin\frac{1}{x}, x\in(0,+\infty) \\
        (\text{C})&\quad f(x)=\frac{1}{x}\sin\frac{1}{x}, x\in(0,+\infty) \\
        (\text{D})&\quad f(x)=\int_0^x\frac{\sin t}{t} dt, x\in(0,2022)
    \end{align*}
    
    \begin{solution}
    \item[(A)] $\lim_{x\to 0^{+}}f(x)=1$, $\lim_{x\to +\infty}=0$均为有限值,故A在区间$(0, +\infty)$有界
    \item[(B)] $\lim_{x\to 0^{+}}f(x)=0$, $\lim_{x\to +\infty}=1$均为有限值,故B在区间$(0, +\infty)$有界
    \item[(C)] $\lim_{x\to 0^{+}}f(x)=+\infty$, $\lim_{x\to +\infty}=0$在0点的极限不为有限值,故C在区间$(0, +\infty)$无界
    \item[(D)] $\lim_{x\to 0^{+}}f(x)=\lim_{x\to 0^{+}}\int_{0}^{x}1dt=0$, 
    $\lim_{x\to 2022^{-}}f(x)=\int_{0}^{2022}\frac{\sin{t}}{t}dt=\text{有限值}$ 故D在区间$(0, 2022)$有界
    \end{solution}
\end{enumerate}

\begin{remark}(导函数与原函数的奇偶性与周期性)
    \item 连续奇函数的所有原函数$\int_{0}^{x}f(t)dt+C$都是偶函数 

    \item 连续偶函数仅有一个原函数$\int_{0}^{x}f(t)\d t$为奇函数
\end{remark}

\begin{enumerate}[label=\arabic*.,start=2]
    \item  (2002,数二)设函数$f(x)$连续,则下列函数中,必为偶函数的是
    \begin{align*}
        (\text{A})&\quad \int_0^x f(t^2) dt \\
        (\text{B})&\quad \int_0^x f^2(t) dt \\
        (\text{C})&\quad \int_0^x t[f(t)-f(-t)] dt \\
        (\text{D})&\quad \int_0^x t[f(t)+f(-t)] dt
    \end{align*}
    
    \begin{solution}
    这种题可以采用奇偶性的定义直接去做,如下面选项A,B的解法,也可以按照上述的函数奇偶性的性质判断
    \item [(A)] 令$F(x) = \int_0^x f(t^2) dt$
    \[
    F(-x)=\int_0^{-x} f(t^2) dt=-\int_0^x f(t^2) dt=-F(x)
    \]
    则A选项是奇函数
    \item [(B)] 
    \[
    F(-x)=\int_0^{-x} f^2(t) dt = -\int_0^x f^2(-t) dt
    \]
    推导不出B的奇偶性
    \item [(C)] $t[f(t)-f(-t)]$是一个偶函数,故C选项是一个奇函数
    \item [(D)] $t[f(t)+f(-t)]$是一个奇函数,故D选项是一个偶函数
    \end{solution}
\end{enumerate}

\subsection{极限的概念}
\begin{definition}[函数极限的定义]
    设函数 $f(x)$ 在点 $x_0$ 的某去心邻域内有定义。若存在常数 $A$,使得对于任意给定的正数 $\epsilon$,总存在正数 $\delta$,使得当 $x$ 满足
    \[
    0 < |x - x_0| < \delta
    \]
    时,必有
    \[
    |f(x) - A| < \epsilon
    \]
    则称 $A$ 为函数 $f(x)$ 当 $x$ 趋近于 $x_0$ 时的极限,记作
    \[
    \lim_{x \to x_0} f(x) = A
    \]
    或
    \[
    f(x) \to A \quad (x \to x_0).
    \]
\end{definition}
\begin{enumerate}[label=\arabic*.,start=3]
    \item  (2014,数三)设$\lim_{n\to\infty} a_n = a$,且$a \neq 0$,则当$n$充分大时有
    \begin{align*}
        (\text{A}) |a_n| > \frac{|a|}{2} \qquad
        (\text{B}) |a_n| < \frac{|a|}{2} \qquad
        (\text{C}) a_n > a - \frac{1}{n} \qquad
        (\text{D}) a_n < a + \frac{1}{n}
    \end{align*}
    
    \begin{solution}
    由数列极限的定义可知当n充分大的时候有$\left|a_n-a\right|<\epsilon$
    \newline 考虑选项C,D,令$\epsilon=\frac{1}{n}$
    则$\left|a_n-a\right|<\frac{1}{n} \implies a-\frac{1}{n}<a_n<a+\frac{1}{n}$
    \end{solution}
\end{enumerate}

\subsection{函数极限的计算}
这一个题型基本上是计算能力的考察,对于常见未定式其实也没必要区分的那么明显,目标都是往最简单$\frac{0}{0}$或者
$\frac{\cdot}{\infty}$模型上面靠,辅助以Taylor公式,拉格朗日中值定理结合夹逼准则来做就可以.
\begin{remark}(类型一 $\frac{0}{0}$型)
\end{remark}

\begin{enumerate}[label=\arabic*.,start=4]
    \item  (2000,数二)若$\lim_{x\to0}\frac{\sin6x+xf(x)}{x^3}=0$,则$\lim_{x\to0}\frac{6+f(x)}{x^2}$为
    \begin{align*}
        (\text{A})&\ 0 \qquad (\text{B})\ 6 \qquad (\text{C})\ 36 \qquad (\text{D})\ \infty
    \end{align*}
    
    \begin{solution}
    这个题第一次见可能想不到,但做多了就一个套路用Taylor就是了.

    $\sin{6x}=6x-36x^2+o(x^3)$,带入题目极限有
    \[
    \lim_{x\to 0}\frac{6x+xf(x)+o(x^3)}{x^3} = \lim_{x\to 0}\frac{6x+xf(x)}{x^3} = 36
    \]
    \end{solution}
    
    \item  (2002,数二)设$y=y(x)$是二阶常系数微分方程$y''+py'+qy=e^{3x}$
    满足初始条件$y(0)=y'(0)=0$的特解,
    则当$x\to0$时,函数$\frac{\ln(1+x^2)}{y(x)}$的极限
    \newline
    \text{(A)}不等于\qquad \text{(B)}等于1\qquad \text{(C)}等于2\qquad \text{(D)}等于3
    
    \begin{solution}
    由微分方程和$y(0)=y'(0)=0$可知$y''(0)=1$,则$y(x)=\frac{1}{2}x^2+o(x^2)$,则
    \[
    \lim_{x\to 0}\frac{\ln(1+x^2)}{y(x)} = \lim_{x\to 0}\frac{x^2}{\frac{1}{2}x^2} = 2
    \]
    \end{solution}
\end{enumerate}

\begin{remark}(类型二 $\frac{\infty}{\infty}$型)
\end{remark}

\begin{enumerate}[label=\arabic*.,start=6]
    \item  (2014,数一、数二、数三)求极限
    \[
    \lim_{x\to \infty}\frac{\int_{1}^{x}\left[t^2(e^{\frac{1}{t}} - 1) - t\right]dt}
    {x^2\ln{(1+\frac{1}{x})}}
    \]
    \begin{solution}
    \begin{align*}
        \lim_{x\to \infty}\frac{\int_{1}^{x}\left[t^2(e^{\frac{1}{t}} - 1) - t\right]dt}
        {x} &= \lim_{x\to \infty} x^2(e^{\frac{1}{x}}-1)-x \\
        & =\lim_{t\to 0}\frac{e^t-1-x}{x^2} \\
        & = \frac{1}{2}
    \end{align*}
    \end{solution}
\end{enumerate}

\begin{remark}(类型三 $0\cdot\infty$型)
\end{remark}

\begin{enumerate}[label=\arabic*.,start=7]
    \item  求极限$\lim_{x\to0^+}\ln(1+x)\ln\left(1+e^{1/x}\right)$
    
    \begin{solution}
    
    \end{solution}
\end{enumerate}

\begin{remark}(类型四 $\infty-\infty$型)
\end{remark}

\begin{enumerate}[label=\arabic*.,start=8]
    \item  求极限$\lim_{x\to\infty}\left(x^3\ln\frac{x+1}{x-1}-2x^2\right)$
    
    \begin{solution}
    【详解】
    \end{solution}
\end{enumerate}

\begin{remark}(类型五 $0^0$与$\infty^0$型)
\end{remark}

\begin{enumerate}[label=\arabic*.,start=9]
    \item  (2010,数三)求极限$\lim_{x\to+\infty}\left(x^{1/x}-1\right)^{1/\ln x}$
    
    \begin{solution}
    【详解】
    \end{solution}
\end{enumerate}

\begin{remark}(类型六 $1^\infty$型)
\end{remark}

\begin{enumerate}[label=\arabic*.,start=10]
    \item  求极限$\lim_{x\to0}\left(\frac{a^x+a^{2x}+\cdots+a^{nx}}{n}\right)^{1/x}\ (a>0,n\in\mathbb{N})$
    
    \begin{solution}
    
    \end{solution}
\end{enumerate}

\subsection{已知极限反求参数}
\begin{remark}(方法)

\end{remark}

\begin{enumerate}[label=\arabic*.,start=11]
    \item  (1998,数二)确定常数$a,b,c$的值,使$\lim_{x\to0}\frac{ax-\sin x}{\int_b^x\frac{\ln(1+t^3)}{t}dt}=c\ (c\neq0)$
    
    \begin{solution}
    【详解】
    \end{solution}
\end{enumerate}

\subsection{无穷小阶的比较}
\begin{remark}(方法)
\end{remark}
\begin{enumerate}[label=\arabic*.,start=12]
    \item  (2002,数二)设函数$f(x)$在$x=0$的某邻域内具有二阶连续导数,且$f(0)\neq0$,$f'(0)\neq0$,$f''(0)\neq0$。证明:存在唯一的一组实数$\lambda_1,\lambda_2,\lambda_3$,使得当$h\to0$时,$\lambda_1f(h)+\lambda_2f(2h)+\lambda_3f(3h)-f(0)$是比$h^2$高阶的无穷小。
    
    \begin{solution}
    【详解】
    \end{solution}
    
    \item  (2006,数二)试确定$A,B,C$的值,使得$e^x(1+Bx+Cx^2)=1+Ax+o(x^3)$,其中$o(x^3)$是当$x\to0$时比$x^3$高阶的无穷小量。
    
    \begin{solution}
    【详解】
    \end{solution}
    
    \item  (2013,数二、数三)当$x\to0$时,$1-\cos x\cdot\cos2x\cdot\cos3x$与$ax^n$为等价无穷小,求$n$与$a$的值。
    
    \begin{solution}
    【详解】
    \end{solution}
\end{enumerate}

\subsection{数列极限的计算}
\begin{remark}(方法)
\end{remark}
\begin{enumerate}[label=\arabic*.,start=15]
    \item  (2011,数一、数二)
    \begin{enumerate}[label=(\roman*)]
        \item 证明:对任意正整数$n$,都有$\frac{1}{n+1}<\ln\left(1+\frac{1}{n}\right)<\frac{1}{n}$
        \item 设$a_n=1+\frac{1}{2}+\cdots+\frac{1}{n}-\ln n\ (n=1,2,\cdots)$,证明数列$\{a_n\}$收敛。
    \end{enumerate}
    
    \begin{solution}
    【详解】
    \end{solution}
    
    \item  (2018,数一、数二、数三)设数列$\{x_n\}$满足:$x_1>0$,$x_ne^{x_{n+1}}=e^{x_n}-1\ (n=1,2,\cdots)$。证明$\{x_n\}$收敛,并求$\lim_{n\to\infty}x_n$。
    
    \begin{solution}
    【详解】
    \end{solution}
    
    \item  (2019,数一、数三)设$a_n=\int_0^1 x^n\sqrt{1-x^2}dx\ (n=0,1,2,\cdots)$。
    \begin{enumerate}[label=(\roman*)]
        \item 证明数列$\{a_n\}$单调减少,且$a_n=\frac{n-1}{n+2}a_{n-2}\ (n=2,3,\cdots)$
        \item 求$\lim_{n\to\infty}\frac{a_n}{a_{n-1}}$
    \end{enumerate}
    
    \begin{solution}
    【详解】
    \end{solution}
    
    \item  (2017,数一、数二、数三)求$\lim_{n\to\infty}\sum_{k=1}^n\frac{k}{n^2}\ln\left(1+\frac{k}{n}\right)$
    
    \begin{solution}
    【详解】
    \end{solution}
\end{enumerate}

\subsection{间断点的判定}

\begin{enumerate}[label=\arabic*.,start=19]
    \item  (2000,数二)设函数$f(x)=\frac{x}{a+e^{bx}}$在$(-\infty,+\infty)$内连续,且$\lim_{x\to-\infty}f(x)=0$,则常数$a,b$满足
    \begin{align*}
        (\text{A})&\ a<0,b<0 \quad (\text{B})\ a>0,b>0 \\
        (\text{C})&\ a\leq0,b>0 \quad (\text{D})\ a\geq0,b<0
    \end{align*}
    
    \begin{solution}
    【详解】
    \end{solution}
\end{enumerate}

\section{一元函数微分学}

\subsection{导数与微分的概念}

\begin{enumerate}[label=\arabic*.]
    \item (2000,数三)设函数$f(x)$在点$x=a$处可导,则函数$|f(x)|$在点$x=a$处不可导的充分条件是
    \begin{align*}
        (A)&\ f(a)=0\ \text{且}\ f'(a)=0 \\
        (B)&\ f(a)=0\ \text{且}\ f'(a)\neq0 \\
        (C)&\ f(a)>0\ \text{且}\ f'(a)>0 \\
        (D)&\ f(a)<0\ \text{且}\ f'(a)<0
    \end{align*}
    
    \begin{solution}
    【详解】
    \end{solution}
    
    \item (2001,数一)设$f(0)=0$,则$f(x)$在$x=0$处可导的充要条件为
    \begin{align*}
        (A)&\ \lim_{h\to0}\frac{1}{h^2}f(1-\cos h)\ \text{存在} \\
        (B)&\ \lim_{h\to0}\frac{1}{h}f(1-e^h)\ \text{存在} \\
        (C)&\ \lim_{h\to0}\frac{1}{h^2}f(h-\sin h)\ \text{存在} \\
        (D)&\ \lim_{h\to0}\frac{1}{h}[f(2h)-f(h)]\ \text{存在}
    \end{align*}
    
    \begin{solution}
    【详解】
    \end{solution}
    
    \item (2016,数一)已知函数$f(x)=\begin{cases}
        x, & x\leq0 \\
        \frac{1}{n}, & \frac{1}{n+1}<x\leq\frac{1}{n},n=1,2,\cdots
    \end{cases}$,则
    \begin{align*}
        (A)&\ x=0\ \text{是}\ f(x)\ \text{的第一类间断点} \\
        (B)&\ x=0\ \text{是}\ f(x)\ \text{的第二类间断点} \\
        (C)&\ f(x)\ \text{在}\ x=0\ \text{处连续但不可导} \\
        (D)&\ f(x)\ \text{在}\ x=0\ \text{处可导}
    \end{align*}
    
    \begin{solution}
    【详解】
    \end{solution}
\end{enumerate}

\subsection{导数与微分的计算}

\begin{remark}[类型一 分段函数求导]
\end{remark}

\begin{enumerate}[label=\arabic*.,start=4]
    \item (1997,数一、数二)设函数$f(x)$连续,$\varphi(x)=\int_0^1 f(xt)dt$,且$\lim_{x\to0}\frac{f(x)}{x}=A$($A$为常数),求$\varphi'(x)$,并讨论$\varphi'(x)$在$x=0$处的连续性。
    
    \begin{solution}
    【详解】
    \end{solution}
\end{enumerate}

\begin{remark}[类型二 复合函数求导]
\end{remark}

\begin{enumerate}[label=\arabic*.,start=5]
    \item (2012,数三)设函数$f(x)=\begin{cases}
        \ln\sqrt{x}, & x\geq1 \\
        2x-1, & x<1
    \end{cases}$,$y=f(f(x))$,求$\left.\frac{dy}{dx}\right|_{x=e}$
    
    \begin{solution}
    【详解】
    \end{solution}
\end{enumerate}

\begin{remark}[类型三 隐函数求导]
\end{remark}

\begin{enumerate}[label=\arabic*.,start=6]
    \item (2007,数二)已知函数$f(u)$具有二阶导数,且$f'(0)=1$,函数$y=y(x)$由方程$y-xe^{y-1}=1$所确定。设$z=f(\ln y-\sin x)$,求$\left.\frac{dz}{dx}\right|_{x=0}$和$\left.\frac{d^2z}{dx^2}\right|_{x=0}$
    
    \begin{solution}
    【详解】
    \end{solution}
\end{enumerate}

\begin{remark}[类型四 反函数求导]
\end{remark}

\begin{enumerate}[label=\arabic*.,start=7]
    \item (2003,数一、数二)设函数$y=y(x)$在$(-\infty,+\infty)$内具有二阶导数,且$y'\neq0$,$x=x(y)$是$y=y(x)$的反函数。
    \begin{enumerate}[label=(\roman*)]
        \item 将$x=x(y)$所满足的微分方程$\frac{d^2x}{dy^2}+(y+\sin x)\left(\frac{dx}{dy}\right)^3=0$变换为$y=y(x)$满足的微分方程
        \item 求变换后的微分方程满足初始条件$y(0)=0$,$y'(0)=\frac{3}{2}$的解
    \end{enumerate}
    
    \begin{solution}
    【详解】
    \end{solution}
\end{enumerate}

\begin{remark}[类型五 参数方程求导]
\end{remark}

\begin{enumerate}[label=\arabic*.,start=8]
    \item (2008,数二)设函数$y=y(x)$由参数方程$\begin{cases}
        x=x(t) \\
        y=\int_0^{t^2}\ln(1+u)du
    \end{cases}$确定,其中$x(t)$是初值问题$\begin{cases}
        \frac{dx}{dt}-2te^{-x}=0 \\
        x|_{t=0}=0
    \end{cases}$的解,求$\frac{d^2y}{dx^2}$
    
    \begin{solution}
    【详解】
    \end{solution}
\end{enumerate}

\begin{remark}[类型六 高阶导数]
\end{remark}

\begin{enumerate}[label=\arabic*.,start=9]
    \item (2015,数二)函数$f(x)=x^2\cdot2^x$在$x=0$处的$n$阶导数$f^{(n)}(0)=$\underline{\quad}
    
    \begin{solution}
    【详解】
    \end{solution}
\end{enumerate}

\subsection{导数应用-切线与法线}

\begin{remark}[类型一 直角坐标表示的曲线]
\end{remark}

\begin{enumerate}[label=\arabic*.,start=10]
    \item  (2000,数二)已知$f(x)$是周期为5的连续函数,它在$x=0$的某个邻域内满足关系式$f(1+\sin x)-3f(1-\sin x)=8x+\alpha(x)$,其中$\alpha(x)$是当$x\to0$时比$x$高阶的无穷小,且$f(x)$在$x=1$处可导,求曲线$y=f(x)$在点$(6,f(6))$处的切线方程。
    
    \begin{solution}
    【详解】
    \end{solution}
\end{enumerate}

\begin{remark}[类型二 参数方程表示的曲线]
\end{remark}

\begin{enumerate}[label=\arabic*.,start=11]
    \item  曲线$\begin{cases}
        x=\int_0^{1-t}e^{-u^2}du \\
        y=t^2\ln(2-t^2)
    \end{cases}$在$(0,0)$处的切线方程为\underline{\quad}
    
    \begin{solution}
    【详解】
    \end{solution}
\end{enumerate}

\begin{remark}[类型三 极坐标表示的曲线]
\end{remark}

\begin{enumerate}[label=\arabic*.,start=12]
    \item  (1997,数一)对数螺线$r=e^\theta$在点$(\frac{\pi}{2},\frac{\pi}{2})$处切线的直角坐标方程为\underline{\quad}
    
    \begin{solution}
    【详解】
    \end{solution}
\end{enumerate}

\subsection{导数应用-渐近线}

\begin{enumerate}[label=\arabic*.,start=13]
    \item  (2014,数一、数二、数三)下列曲线中有渐近线的是
    \begin{align*}
        (A)&\ y=x+\sin x \quad (B)\ y=x^2+\sin x \\
        (C)&\ y=x+\sin\frac{1}{x} \quad (D)\ y=x^2+\sin\frac{1}{x}
    \end{align*}
    
    \begin{solution}
    【详解】
    \end{solution}
    
    \item  (2007,数一、数二、数三)曲线$y=\frac{1}{x}+\ln(1+e^x)$渐近线的条数为
    \begin{align*}
        (A)\ 0 \quad (B)\ 1 \quad (C)\ 2 \quad (D)\ 3
    \end{align*}
    
    \begin{solution}
    【详解】
    \end{solution}
\end{enumerate}

\subsection{导数应用-曲率}

\begin{enumerate}[label=\arabic*.,start=15]
    \item  (2014,数二)曲线$\begin{cases}
        x=t^2+7 \\
        y=t^2+4t+1
    \end{cases}$对应于$t=1$的点处的曲率半径是
    \begin{align*}
        (A)\ \frac{\sqrt{10}}{50} \quad (B)\ \frac{\sqrt{10}}{100} \quad (C)\ 10\sqrt{10} \quad (D)\ 5\sqrt{10}
    \end{align*}
    
    \begin{solution}
    【详解】
    \end{solution}
\end{enumerate}

\subsection{导数应用-极值与最值}

\begin{enumerate}[label=\arabic*.,start=17]
    \item  (2000,数二)设函数$f(x)$满足关系式$f''(x)+[f'(x)]^2=x$,且$f'(0)=0$,则
    \begin{align*}
        (A)&\ f(0)\ \text{是}\ f(x)\ \text{的极大值} \\
        (B)&\ f(0)\ \text{是}\ f(x)\ \text{的极小值} \\
        (C)&\ \text{点}(0,f(0))\ \text{是曲线}\ y=f(x)\ \text{的拐点} \\
        (D)&\ f(0)\ \text{不是}\ f(x)\ \text{的极值,点}(0,f(0))\ \text{也不是曲线}\ y=f(x)\ \text{的拐点}
    \end{align*}
    
    \begin{solution}
    【详解】
    \end{solution}
    
    \item  (2010,数一、数二)求函数$f(x)=\int_1^{x^2}(x^2-t)e^{-t^2}dt$的单调区间与极值
    
    \begin{solution}
    【详解】
    \end{solution}
    
    \item  (2014,数二)已知函数$y=y(x)$满足微分方程$x^2+y^2y'=1-y'$,且$y(2)=0$,求$y(x)$的极大值与极小值
    
    \begin{solution}
    【详解】
    \end{solution}
\end{enumerate}

\subsection{导数应用-凹凸性与拐点}

\begin{enumerate}[label=\arabic*.,start=20]
    \item  (2011,数一)曲线$y=(x-1)(x-2)^2(x-3)^3(x-4)^4$的拐点是
    \begin{align*}
        (A)\ (1,0) \quad (B)\ (2,0) \quad (C)\ (3,0) \quad (D)\ (4,0)
    \end{align*}
    
    \begin{solution}
    【详解】
    \end{solution}
\end{enumerate}

\subsection{导数应用-证明不等式}

\begin{enumerate}[label=\arabic*.,start=21]
    \item  (2017,数一、数三)设函数$f(x)$可导,且$f(x)f'(x)>0$,则
    \begin{align*}
        (A)\ f(1)>f(-1) \quad (B)\ f(1)<f(-1) \\
        (C)\ |f(1)|>|f(-1)| \quad (D)\ |f(1)|<|f(-1)|
    \end{align*}
    
    \begin{solution}
    【详解】
    \end{solution}
    
    \item  (2015,数二)已知函数$f(x)$在区间$[a,+\infty)$上具有二阶导数,$f(a)=0$,$f'(x)>0$,$f''(x)>0$。设$b>a$,曲线$y=f(x)$在点$(b,f(b))$处的切线与$x$轴的交点是$(x_0,0)$,证明$a<x_0<b$。
    
    \begin{solution}
    【详解】
    \end{solution}
\end{enumerate}

\subsection{导数应用-求方程的根}

\begin{enumerate}[label=\arabic*.,start=23]
    \item  (2003,数二)讨论曲线$y=4\ln x+k$与$y=4x+\ln^4 x$的交点个数。
    
    \begin{solution}
    【详解】
    \end{solution}
    
    \item  (2015,数二)已知函数$f(x)=\int_x^1\sqrt{1+t^2}dt+\int_1^{x^2}\sqrt{1+t}dt$,求$f(x)$零点的个数。
    
    \begin{solution}
    【详解】
    \end{solution}
\end{enumerate}

\subsection{微分中值定理证明题}

\begin{remark}[类型一 证明含有一个点的等式]
\end{remark}

\begin{enumerate}[label=\arabic*.,start=25]
    \item  (2013,数一、数二)设奇函数$f(x)$在$[-1,1]$上具有二阶导数,且$f(1)=1$。证明:
    \begin{enumerate}[label=(\roman*)]
        \item 存在$\xi\in(0,1)$,使得$f'(\xi)=1$;
        \item 存在$\eta\in(-1,1)$,使得$f''(\eta)+f'(\eta)=1$。
    \end{enumerate}
    
    \begin{solution}
    【详解】
    \end{solution}
    
    \item  设函数$f(x)$在$[0,1]$上连续,在$(0,1)$内可导,$f(1)=0$,证明:存在$\xi\in(0,1)$,使得$(2\xi+1)f(\xi)+\xi f'(\xi)=0$。
    
    \begin{solution}
    【详解】
    \end{solution}
\end{enumerate}

\begin{remark}[类型二 证明含有两个点的等式]
\end{remark}

\begin{enumerate}[label=\arabic*.,start=27]
    \item  设$f(x)$在$[0,1]$上连续,在$(0,1)$内可导,且$f(0)=0$,$f(1)=1$。证明:
    \begin{enumerate}[label=(\roman*)]
        \item 存在两个不同的点$\xi_1,\xi_2\in(0,1)$,使得$f'(\xi_1)+f'(\xi_2)=2$;
        \item 存在$\xi,\eta\in(0,1)$,使得$\eta f'(\xi)=f(\eta)f'(\eta)$。
    \end{enumerate}
    
    \begin{solution}
    【详解】
    \end{solution}
\end{enumerate}

\begin{remark}[类型三 证明含有高阶导数的等式或不等式]
\end{remark}

\begin{enumerate}[label=\arabic*.,start=28]
    \item  (2019,数二)已知函数$f(x)$在$[0,1]$上具有二阶导数,且$f(0)=0$,$f(1)=1$,$\int_0^1 f(x)dx=1$。证明:
    \begin{enumerate}[label=(\roman*)]
        \item 存在$\xi\in(0,1)$,使得$f'(\xi)=0$;
        \item 存在$\eta\in(0,1)$,使得$f''(\eta)<-2$。
    \end{enumerate}
    
    \begin{solution}
    【详解】
    \end{solution}
\end{enumerate}

\section{一元函数积分学}
\subsection{ 定积分的概念}

\begin{enumerate}[label=\arabic*.]
    \item 例1 (2007,数一、数二、数三)如图,连续函数$y=f(x)$在区间[-3,-2],[2,3]上的图形分别是直径为1的上、下半圆周,在区间[-2,0],[0,2]的图形分别是直径为2的下、上半圆周.
    设$F(x)=\int_0^x f(t) dt$,则下列结论正确的是:
    \begin{align*}
        (A) F(3)=-\frac{3}{4} F(-2)
    \end{align*}
    
    \begin{solution}
    【详解】
    \end{solution}
    
    \item 例2 (2009,数三)使不等式$\int_1^x\frac{\sin t}{t} dt>\ln x$成立的$x$的范围是
    \begin{align*}
        (A)\ (0,1)\quad(B)\left(1,\frac{\pi}{2}\right)\quad(C)\left(\frac{\pi}{2},\pi\right)\quad(D)(\pi,+\infty)
    \end{align*}
    
    \begin{solution}
    【详解】
    \end{solution}
    
    \item 例3 (2003,数二)设$I_1=\int_0^{\frac{\pi}{4}}\frac{\tan x}{x} dx, I_2=\int_0^{\frac{\pi}{4}}\frac{x}{\tan x} dx$,则
    \begin{align*}
        (A) I_1>I_2>1\quad(B) 1>I_1>I_2 \\
        (C) I_2>I_1>1\quad(D) 1>I_2>I_1
    \end{align*}
    
    \begin{solution}
    【详解】
    \end{solution}
\end{enumerate}

\subsection{ 不定积分的计算}

\begin{enumerate}[label=\arabic*.,start=4]
    \item 例5 (2009,数二、数三)计算不定积分$\int\frac{1}{1+\sqrt{\frac{1+x}{x}}}dx(x>0)$
    
    \begin{solution}
    【详解】
    \end{solution}
    
    \item 例6 求$\int\frac{1}{1+\sin x+\cos x} dx$
    
    \begin{solution}
    【详解】
    \end{solution}
\end{enumerate}

\subsection{ 定积分的计算}

\begin{enumerate}[label=\arabic*.,start=6]
    \item 例7 (2013,数一)计算$\int_0^1\frac{f(x)}{\sqrt{x}} dx$,其中$f(x)=\int_1^x\frac{\ln(t+1)}{t} dt$
    
    \begin{solution}
    【详解】
    \end{solution}
    
    \item 例8 求下列积分:
    \begin{align*}
        (1)\ \int_0^{\frac{\pi}{2}}\frac{1}{1+(\tan x)^{\sqrt{2}}} dx
    \end{align*}
    
    \begin{solution}
    【详解】
    \end{solution}
    
    \item 例9 求$\int_0^{\frac{\pi}{4}}\ln(1+\tan x) dx$
    
    \begin{solution}
    【详解】
    \end{solution}
\end{enumerate}

\subsection{ 反常积分的计算}

\begin{enumerate}[label=\arabic*.,start=9]
    \item 例10 (1998,数二)计算积分(题目内容缺失)
    
    \begin{solution}
    【详解】
    \end{solution}
\end{enumerate}

\subsection{ 反常积分敛散性的判定}

\begin{enumerate}[label=\arabic*.,start=10]
    \item 例11 (2016,数一)若反常积分$\int_0^{+\infty}\frac{1}{x^a(1+x)^b} dx$收敛,则
    \begin{align*}
        (A)\ a<1且\ b>1 \\
        (B)\ a>1且\ b>1 \\
        (C)\ a<1且\ a+b>1 \\
        (D)\ a>1且\ a+b>1
    \end{align*}
    
    \begin{solution}
    【详解】
    \end{solution}
    
    \item 例12 (2010,数一、数二)设$m,n$均为正整数,则反常积分$\int_0^1\frac{\sqrt[n]{\ln^2(1-x)}}{\sqrt[n]{x}} dx$的收敛性
    \begin{align*}
        (A)\ 仅与\ m\ 的取值有关 \\
        (B)\ 仅与\ n\ 的取值有关 \\
        (C)\ 与\ m,n\ 的取值都有关 \\
        (D)\ 与\ m,n\ 的取值都无关
    \end{align*}
    
    \begin{solution}
    【详解】
    \end{solution}
\end{enumerate}

\subsection{ 变限积分函数}

\begin{enumerate}[label=\arabic*.,start=12]
    \item 例13 (2013,数二)设函数$f(x)=\begin{cases}
        \sin x, & 0\leq x<\pi \\
        2, & \pi\leq x\leq 2\pi
    \end{cases}$,$F(x)=\int_0^x f(t) dt$,则
    \begin{align*}
        (A)\ x=\pi\ 是函数\ F(x)\ 的跳跃间断点 \\
        (B)\ x=\pi\ 是函数\ F(x)\ 的可去间断点 \\
        (C)\ F(x)\ 在\ x=\pi\ 处连续但不可导 \\
        (D)\ F(x)\ 在\ x=\pi\ 处可导
    \end{align*}
    
    \begin{solution}
    【详解】
    \end{solution}
    
    \item 例14 (2016,数二)已知函数$f(x)$在$[0,3\pi]$上连续,在$(0,3\pi)$内是函数的一个原函数,且$f(0)=0$.
    \begin{enumerate}[label=(\roman*)]
        \item 求$f(x)$在区间$[0,\frac{3\pi}{2}]$上的平均值;
        \item 证明$f(x)$在区间$[0,\frac{3\pi}{2}]$内存在唯一零点.
    \end{enumerate}
    
    \begin{solution}
    【详解】
    \end{solution}
\end{enumerate}

\subsection{ 定积分应用求面积}

\begin{enumerate}[label=\arabic*.,start=14]
    \item 例15 (2019,数一、数二、数三)求曲线$y=e^{-x}\sin x(x\geq 0)$与$x$轴之间图形的面积.
    
    \begin{solution}
    【详解】
    \end{solution}
\end{enumerate}

\subsection{ 定积分应用求体积}

\begin{enumerate}[label=\arabic*.,start=15]
    \item 例16 (2003,数一)过原点作曲线$y=\ln x$的切线,该切线与曲线$y=\ln x$及$x$轴围成平面图形$D$.
    \begin{enumerate}[label=(\roman*)]
        \item 求$D$的面积$A$;
        \item 求$D$绕直线$x=e$旋转一周所得旋转体的体积$V$.
    \end{enumerate}
    
    \begin{solution}
    【详解】
    \end{solution}
    
    \item 例17 (2014,数二)已知函数$f(x, y)$满足$\frac{\partial f}{\partial y}=2(y+1)$,且$f(y, y)=(y+1)^2-(2-y)\ln y$,求曲线$f(x, y)=0$所围图形绕直线$y=-1$旋转所成旋转体的体积.
    
    \begin{solution}
    【详解】
    \end{solution}
\end{enumerate}

\subsection{ 定积分应用求弧长}

\begin{enumerate}[label=\arabic*.,start=17]
    \item 例18 求心形线$r=a(1+\cos\theta)(a>0)$的全长.
    
    \begin{solution}
    【详解】
    \end{solution}
\end{enumerate}

\subsection{ 定积分应用求侧面积}

\begin{enumerate}[label=\arabic*.,start=18]
    \item 例19 (2016,数二)设$D$是由曲线$y=\sqrt{1-x^2}(0\leq x\leq 1)$与$x=\cos^3 t$围成的平面区域,求$D$绕$x$轴旋转一周所得旋转体的体积和表面积.
    
    \begin{solution}
    【详解】
    \end{solution}
\end{enumerate}

\subsection{一 定积分物理应用}

\begin{enumerate}[label=\arabic*.,start=19]
    \item 例20 (2020,数二)设边长为$2a$等腰直角三角形平板铅直地沉没在水中,且斜边与水面相齐,设重力加速度为$g$,水密度为$\rho$,则该平板一侧所受的水压力为
    
    \begin{solution}
    【详解】
    \end{solution}
\end{enumerate}

\subsection{二 证明含有积分的等式或不等式}

\begin{enumerate}[label=\arabic*.,start=20]
    \item 例21 (2000,数二)设函数$S(x)=\int_0^x|\cos t| dt$.
    \begin{enumerate}[label=(\roman*)]
        \item 当$n$为正整数,且$n\pi\leq x<(n+1)\pi$时,证明$2n\leq S(x)<2(n+1)$;
        \item 求$\lim_{x\to+\infty}\frac{S(x)}{x}$
    \end{enumerate}
    
    \begin{solution}
    【详解】
    \end{solution}
    
    \item 例22 (2014,数二、数三)设函数$f(x), g(x)$在区间$[a, b]$上连续,且$f(x)$单调增加,$0\leq g(x)\leq 1$.
    证明:
    \begin{enumerate}[label=(\roman*)]
        \item $0\leq\int_a^x g(t) dt\leq x-a, x\in[a, b]$;
        \item $\int_a^{a+\int_a^b g(t) dt} f(x) dx\leq\int_a^b f(x) g(x) dx$.
    \end{enumerate}
    
    \begin{solution}
    【详解】
    \end{solution}
\end{enumerate}

\section{常微分方程}
\begin{enumerate}[label=\arabic*.]
    \item 例1 (1998,数一、数二)已知函数$y=y(x)$在任意点$x$处的增量$\Delta y=\frac{y\Delta x}{1+x^2}+\alpha$,其中$\alpha$是$\Delta x$的高阶无穷小,$y(0)=\pi$,则$y(1)$等于
    \begin{align*}
        (A)\ 2\pi \quad (B)\ \pi \quad (C)\ e^{\frac{\pi}{4}} \quad (D)\ \pi e^{\frac{\pi}{4}}
    \end{align*}
    
    \begin{solution}
    【详解】
    \end{solution}
    
    \item 例2 (2002,数二)已知函数$f(x)$在$(0,+\infty)$内可导,$f(x)>0$,$\lim_{x\to+\infty}f(x)=1$,且满足
    \begin{align*}
        \lim_{h\to0}\left(\frac{f(x+hx)}{f(x)}\right)^{\frac{1}{h}}=e^{\frac{1}{x}}
    \end{align*}
    求$f(x)$。
    
    \begin{solution}
    【详解】
    \end{solution}
\end{enumerate}

\subsection{ 一阶微分方程的解法}

\begin{remark}[类型一 可分离变量]
\end{remark}

\begin{enumerate}[label=\arabic*.,start=3]
    \item 例3 (1999,数二)求初值问题
    \begin{align*}
        \begin{cases}
            (y+\sqrt{x^2+y^2})dx-xdy=0 & (x>0) \\
            y|_{x=1}=0
        \end{cases}
    \end{align*}
    
    \begin{solution}
    【详解】
    \end{solution}
\end{enumerate}

\begin{remark}[类型二 一阶齐次]
\end{remark}

\begin{enumerate}[label=\arabic*.,start=4]
    \item 例4 (2010,数二、数三)设$y_1,y_2$是一阶线性非齐次微分方程$y'+p(x)y=q(x)$的两个特解。若常数$\lambda,\mu$使$\lambda y_1+\mu y_2$是该方程的解,$\lambda y_1-\mu y_2$是该方程对应的齐次方程的解,则
    \begin{align*}
        (A)\ \lambda=\frac{1}{2},\ \mu=\frac{1}{2} \quad (C)\ \lambda=\frac{2}{3},\ \mu=\frac{1}{3}
    \end{align*}
    
    \begin{solution}
    【详解】
    \end{solution}
\end{enumerate}

\begin{remark}[类型三 一阶线性]
\end{remark}

\begin{enumerate}[label=\arabic*.,start=5]
    \item 例5 (2018,数一)已知微分方程$y'+y=f(x)$,其中$f(x)$是$\mathbb{R}$上的连续函数。
    \begin{enumerate}[label=(\roman*)]
        \item 若$f(x)=x$,求方程的通解;
        \item 若$f(x)$是周期为$T$的函数,证明:方程存在唯一的以$T$为周期的解。
    \end{enumerate}
    
    \begin{solution}
    【详解】
    \end{solution}
\end{enumerate}

\begin{remark}[类型四 伯努利方程(数一掌握)]
\end{remark}

\begin{enumerate}[label=\arabic*.,start=6]
    \item 例6 求解微分方程$y'=\frac{y}{x}+\sqrt{\frac{y^2}{x^2}-1}$。
    
    \begin{solution}
    【详解】
    \end{solution}
\end{enumerate}

\begin{remark}[类型五 全微分方程(数一掌握)]
\end{remark}

\begin{enumerate}[label=\arabic*.,start=7]
    \item 例7 求解下列微分方程:
    \begin{align*}
        (1)\ &(2xe^y+3x^2-1)dx+(x^2e^y-2y)dy=0; \\
        (2)\ &\frac{2x}{y^3}dx+\frac{y^2-3x^2}{y^4}dy=0.
    \end{align*}
    
    \begin{solution}
    【详解】
    \end{solution}
\end{enumerate}

\subsection{二阶常系数线性微分方程}

\begin{enumerate}[label=\arabic*.,start=8]
    \item 例8 (2017,数二)微分方程$y''-4y'+8y=e^{2x}(1+\cos2x)$的特解可设为$y^*=$
    \begin{align*}
        (A)\ Ae^{2x}+e^{2x}(B\cos2x+C\sin2x) \\
        (B)\ Axe^{2x}+e^{2x}(B\cos2x+C\sin2x) \\
        (C)\ Ae^{2x}+xe^{2x}(B\cos2x+C\sin2x) \\
        (D)\ Axe^{2x}+xe^{2x}(B\cos2x+C\sin2x)
    \end{align*}
    
    \begin{solution}
    【详解】
    \end{solution}
    
    \item 例9 (2015,数一)设$y=\frac{1}{2}e^{2x}+(x-\frac{1}{3})e^x$是二阶常系数非齐次线性微分方程$y''+ay'+by=ce^x$的一个特解,则
    \begin{align*}
        (A)\ a=-3,b=2,c=-1 \\
        (B)\ a=3,b=2,c=-1 \\
        (C)\ a=-3,b=2,c=1 \\
        (D)\ a=3,b=2,c=1
    \end{align*}
    
    \begin{solution}
    【详解】
    \end{solution}
    
    \item 例10 (2016,数二)已知$y_1(x)=e^x$,$y_2(x)=u(x)e^x$是二阶微分方程$(2x-1)y''-(2x+1)y'+2y=0$的两个解。若$u(-1)=e$,$u(0)=-1$,求$u(x)$,并写出该微分方程的通解。
    
    \begin{solution}
    【详解】
    \end{solution}
    
    \item 例11 (2016,数一)设函数$y(x)$满足方程$y''+2y'+ky=0$,其中$0<k<1$。
    \begin{enumerate}[label=(\roman*)]
        \item 证明反常积分$\int_0^{+\infty}y(x)dx$收敛;
        \item 若$y(0)=1$,$y'(0)=1$,求$\int_0^{+\infty}y(x)dx$的值。
    \end{enumerate}
    
    \begin{solution}
    【详解】
    \end{solution}
\end{enumerate}

\subsection{ 高阶常系数线性齐次微分方程}

\begin{enumerate}[label=\arabic*.,start=12]
    \item 例12 求解微分方程$y^{(4)}-3y''-4y=0$。
    
    \begin{solution}
    【详解】
    \end{solution}
\end{enumerate}

\subsection{ 二阶可降阶微分方程}

\begin{remark}[方法 数一、数二掌握 数三大纲不要求]
\end{remark}

\begin{enumerate}[label=\arabic*.,start=13]
    \item 例13 求微分方程$y''(x+y'^2)=y'$满足初始条件$y(1)=y'(1)=1$的特解。
    
    \begin{solution}
    【详解】
    \end{solution}
\end{enumerate}

\subsection{ 欧拉方程}

\begin{remark}[方法 数一掌握 数二、数三大纲不要求]
\end{remark}

\begin{enumerate}[label=\arabic*.,start=14]
    \item 例14 求解微分方程$x^2y''+xy'+y=2\sin\ln x$。
    
    \begin{solution}
    【详解】
    \end{solution}
\end{enumerate}

\subsection{ 变量代换求解二阶变系数线性微分方程}

\begin{enumerate}[label=\arabic*.,start=17]
    \item 例17 (2005,数二)用变量代换$x=\cos t(0<t<\pi)$化简微分方程$(1-x^2)y''-xy'+y=0$,并求其满足$y|_{x=0}=1$,$y'|_{x=0}=2$的特解。
    
    \begin{solution}
    【详解】
    \end{solution}
\end{enumerate}

\subsection{ 微分方程综合题}

\begin{remark}[类型一 综合导数应用]
\end{remark}

\begin{enumerate}[label=\arabic*.,start=18]
    \item 例18 (2001,数二)设$L$是一条平面曲线,其上任意一点$P(x,y)(x>0)$到坐标原点的距离,恒等于该点处的切线在$y$轴上的截距,且$L$经过点$(\frac{1}{2},0)$。求曲线$L$的方程。
    
    \begin{solution}
    【详解】
    \end{solution}
\end{enumerate}

\begin{remark}[类型二 综合定积分应用]
\end{remark}

\begin{enumerate}[label=\arabic*.,start=19]
    \item 例19 (2009,数三)设曲线$y=f(x)$,其中$f(x)$是可导函数,且$f(x)>0$。已知曲线$y=f(x)$与直线$y=0$,$x=1$及$x=t(t>1)$所围成的曲边梯形绕$x$轴旋转一周所得的立体体积值是该曲边梯形面积值的$\pi t$倍,求该曲线的方程。
    
    \begin{solution}
    【详解】
    \end{solution}
\end{enumerate}

\begin{remark}[类型三 综合变限积分]
\end{remark}

\begin{enumerate}[label=\arabic*.,start=20]
    \item 例20 (2016,数三)设函数$f(x)$连续,且满足$\int_0^x f(x-t)dt=\int_0^x(x-t)f(t)dt+e^{-x}-1$,求$f(x)$。
    
    \begin{solution}
    【详解】
    \end{solution}
\end{enumerate}

\begin{remark}[类型四 综合多元复合函数]
\end{remark}

\begin{enumerate}[label=\arabic*.,start=21]
    \item 例21 (2014,数一、数二、数三)设函数$f(u)$具有二阶连续导数,$z=f(e^x\cos y)$满足
    \begin{align*}
        \frac{\partial^2 z}{\partial x^2}+\frac{\partial^2 z}{\partial y^2}=(4z+e^x\cos y)e^{2x}
    \end{align*}
    若$f(0)=0$,$f'(0)=0$,求$f(u)$的表达式。
    
    \begin{solution}
    【详解】
    \end{solution}
\end{enumerate}

\begin{remark}[类型五 综合重积分]
\end{remark}

\begin{enumerate}[label=\arabic*.,start=22]
    \item 例22 (2011,数三)设函数$f(x)$在区间$[0,1]$上具有连续导数,$f(0)=1$,且满足
    \begin{align*}
        \iint_{D_t} f'(x+y)dxdy=\iint_{D_t} f(t)dxdy
    \end{align*}
    其中$D_t=\{(x,y)|0\leq y\leq t-x,0\leq x\leq t\}(0<t\leq1)$,求$f(x)$的表达式。
    
    \begin{solution}
    【详解】
    \end{solution}
\end{enumerate}

\section{多元函数微分学}
\subsection{多元函数的概念}

\begin{enumerate}[label=\arabic*.]
    \item 例1 求下列重极限:
    \begin{align*}
        (1)\ \lim_{\substack{x\to 0\\ y\to 0}}\frac{x^\alpha y^\beta}{x^2+y^2}\quad (\alpha\geq0,\beta\geq0); \\
        (2)\ \lim_{\substack{x\to 0\\ y\to 0}}\frac{xy(x^{2}-y^{2})}{x^{2}+y^{2}};
    \end{align*}
    
    \begin{solution}
    【详解】
    \end{solution}
    
    \item 例2 (2012,数一)如果函数$f(x,y)$在点$(0,0)$处连续,那么下列命题正确的是
    \begin{align*}
        (A)\ \text{若极限}\lim_{\substack{x\to 0\\ y\to 0}}\frac{f(x,y)}{|x|+|y|}\text{存在,则}f(x,y)\text{在点}(0,0)\text{处可微} \\
        (B)\ \text{若极限}\lim_{\substack{x\to 0\\ y\to 0}}\frac{f(x,y)}{x^{2}+y^{2}}\text{存在,则}f(x,y)\text{在点}(0,0)\text{处可微} \\
        (C)\ \text{若}f(x,y)\text{在点}(0,0)\text{处可微,则极限}\lim_{\substack{x\to 0\\ y\to 0}}\frac{f(x,y)}{|x|+|y|}\text{存在} \\
        (D)\ \text{若}f(x,y)\text{在点}(0,0)\text{处可微,则极限}\lim_{\substack{x\to 0\\ y\to 0}}\frac{f(x,y)}{x^{2}+y^{2}}\text{存在}
    \end{align*}
    
    \begin{solution}
    【详解】
    \end{solution}
    
    \item 例3 (2012,数三)设连续函数$z=f(x,y)$满足
    \begin{align*}
        \lim_{\substack{x\to 0\\ y\to 1}}\frac{f(x,y)-2x+y-2}{\sqrt{x^2+(y-1)^2}}=0
    \end{align*}
    则$\left.dz\right|_{(0,1)}=$
    
    \begin{solution}
    【详解】
    \end{solution}
\end{enumerate}

\subsection{多元复合函数求偏导数与全微分}

\begin{enumerate}[label=\arabic*.,start=4]
    \item 例4 (2021,数一、数二、数三)设函数$f(x,y)$可微,且
    \begin{align*}
        f(x+1,e^x)=x(x+1)^2, \\
        f(x,x^2)=2x^2\ln x
    \end{align*}
    则$df(1,1)=$
    \begin{align*}
        (A)\ dx+dy \quad (B)\ dx-dy \quad (C)\ dy
    \end{align*}
    
    \begin{solution}
    【详解】
    \end{solution}
    
    \item 例5 (2011,数一、数二)设$z=f(xy,yg(x))$,其中函数$f$具有二阶连续偏导数,函数$g(x)$可导,且在$x=1$处取得极值$g(1)=1$,求$\left.\frac{\partial^2 z}{\partial x\partial y}\right|_{x=1,y=1}$。
    
    \begin{solution}
    【详解】
    \end{solution}
\end{enumerate}

\subsection{多元隐函数求偏导数与全微分}

\begin{enumerate}[label=\arabic*.,start=6]
    \item 例6 (2005,数一)设有三元方程$xy-z\ln y+e^{xz}=1$,根据隐函数存在定理,存在点$(0,1,1)$的一个邻域,在此邻域内该方程
    \begin{align*}
        (A)\ \text{只能确定一个具有连续偏导数的隐函数}z=z(x,y) \\
        (B)\ \text{可确定两个具有连续偏导数的隐函数}x=x(y,z)\text{和}z=z(x,y) \\
        (C)\ \text{可确定两个具有连续偏导数的隐函数}y=y(x,z)\text{和}z=z(x,y) \\
        (D)\ \text{可确定两个具有连续偏导数的隐函数}x=x(y,z)\text{和}y=y(x,z)
    \end{align*}
    
    \begin{solution}
    【详解】
    \end{solution}
    
    \item 例7 (1999,数一)设$y=y(x),z=z(x)$是由方程$z=xf(x+y)$和$F(x,y,z)=0$所确定的函数,其中$f$和$F$分别具有一阶连续导数和一阶连续偏导数,求$\frac{dz}{dx}$。
    
    \begin{solution}
    【详解】
    \end{solution}
\end{enumerate}

\subsection{变量代换化简偏微分方程}

\begin{enumerate}[label=\arabic*.,start=8]
    \item 例8 (2010,数二)设函数$u=f(x,y)$具有二阶连续偏导数,且满足等式
    \begin{align*}
        4\frac{\partial^2 u}{\partial x^2}+12\frac{\partial^2 u}{\partial x\partial y}+5\frac{\partial^2 u}{\partial y^2}=0
    \end{align*}
    确定$a,b$的值,使等式在变换$\xi=x+ay,\eta=x+by$下简化为$\frac{\partial^2 u}{\partial \xi\partial \eta}=0$。
    
    \begin{solution}
    【详解】
    \end{solution}
\end{enumerate}

\subsection{求无条件极值}

\begin{enumerate}[label=\arabic*.,start=9]
    \item 例9 (2003,数一)已知函数$f(x,y)$在点$(0,0)$的某个邻域内连续,且
    \begin{align*}
        \lim_{\substack{x\to 0\\ y\to 0}}\frac{f(x,y)-xy}{(x^2+y^2)^2}=1
    \end{align*}
    则
    \begin{align*}
        (A)\ \text{点}(0,0)\text{不是}f(x,y)\text{的极值点} \\
        (B)\ \text{点}(0,0)\text{是}f(x,y)\text{的极大值点} \\
        (C)\ \text{点}(0,0)\text{是}f(x,y)\text{的极小值点} \\
        (D)\ \text{根据所给条件无法判别点}(0,0)\text{是否为}f(x,y)\text{的极值点}
    \end{align*}
    
    \begin{solution}
    【详解】
    \end{solution}
    
    \item 例10 (2004,数一)设$z=z(x,y)$是由$x^2-6xy+10y^2-2yz-z^2+18=0$确定的函数,求$z=z(x,y)$的极值点和极值。
    
    \begin{solution}
    【详解】
    \end{solution}
\end{enumerate}

\subsection{求条件极值(边界最值)}

\begin{enumerate}[label=\arabic*.,start=11]
    \item 例11 (2006,数一、数二、数三)设$f(x,y)$与$\varphi(x,y)$均为可微函数,且$\varphi_y'(x,y)\neq 0$。已知$(x_0,y_0)$是$f(x,y)$在约束条件$\varphi(x,y)=0$下的一个极值点,下列选项正确的是
    \begin{align*}
        (A)\ \text{若}f_x'(x_0,y_0)=0\text{,则}f_y'(x_0,y_0)=0 \\
        (B)\ \text{若}f_x'(x_0,y_0)=0\text{,则}f_y'(x_0,y_0)\neq 0 \\
        (C)\ \text{若}f_x'(x_0,y_0)\neq 0\text{,则}f_y'(x_0,y_0)=0 \\
        (D)\ \text{若}f_x'(x_0,y_0)\neq 0\text{,则}f_y'(x_0,y_0)\neq 0
    \end{align*}
    
    \begin{solution}
    【详解】
    \end{solution}
    
    \item 例12 (2013,数二)求曲线$x^3-xy+y^3=1(x\geq 0,y\geq 0)$上的点到坐标原点的最长距离与最短距离。
    
    \begin{solution}
    【详解】
    \end{solution}
    
    \item 例13 (2014,数二)设函数$u(x,y)$在有界闭区域$D$上连续,在$D$的内部具有二阶连续偏导数,且满足$\frac{\partial^2 u}{\partial x\partial y}\neq 0$及$\frac{\partial^2 u}{\partial x^2}+\frac{\partial^2 u}{\partial y^2}=0$,则
    \begin{align*}
        (A)\ u(x,y)\text{的最大值和最小值都在}D\text{的边界上取得} \\
        (B)\ u(x,y)\text{的最大值和最小值都在}D\text{的内部取得} \\
        (C)\ u(x,y)\text{的最大值在}D\text{的内部取得,最小值在}D\text{的边界上取得} \\
        (D)\ u(x,y)\text{的最小值在}D\text{的内部取得,最大值在}D\text{的边界上取得}
    \end{align*}
    
    \begin{solution}
    【详解】
    \end{solution}
    
    \item 例14 (2005,数二)已知函数$z=f(x,y)$的全微分$dz=2xdx-2ydy$,且$f(1,1)=2$,求$f(x,y)$在椭圆域$D=\{(x,y)|x^2+\frac{y^2}{4}\leq 1\}$上的最大值和最小值。
    
    \begin{solution}
    【详解】
    \end{solution}
\end{enumerate}

\section{二重积分}
\subsection{二重积分的概念}

\begin{enumerate}[label=\arabic*.]
    \item 例1 (2010,数一、数二) 
    \begin{align*}
        \lim_{n\rightarrow\infty}\sum_{i=1}^n\sum_{j=1}^n\frac{n}{(n+i)(n^2+j^2)}=
    \end{align*}
    \begin{align*}
        (A)\int_0^1 dx\int_0^x\frac{1}{(1+x)(1+y^2)}dy \quad (B)\int_0^1 dx\int_0^x\frac{1}{(1+x)(1+y)}dy \\
        (C)\int_0^1 dx\int_0^1\frac{1}{(1+x)(1+y)}dy \quad (D)\int_0^1 dx\int_0^1\frac{1}{(1+x)(1+y^2)}dy
    \end{align*}
    
    \begin{solution}
    【详解】
    \end{solution}
    
    \item 例2 (2016,数三)设$J_i=\iint_{D_i}\sqrt[3]{x-y}dxdy(i=1,2,3)$,其中
    \begin{align*}
        D_1=\{(x,y)|0\leq x\leq 1,0\leq y\leq 1\}, \\
        D_2=\{(x,y)|0\leq x\leq 1,0\leq y\leq \sqrt{x}\}, \\
        D_3=\{(x,y)|0\leq x\leq 1,x^2\leq y\leq 1\},
    \end{align*}
    则
    \begin{align*}
        (A)\ J_1<J_2<J_3 \quad (B)\ J_3<J_1<J_2 \\
        (C)\ J_2<J_3<J_1 \quad (D)\ J_2<J_1<J_3
    \end{align*}
    
    \begin{solution}
    【详解】
    \end{solution}
\end{enumerate}

\subsection{交换积分次序}

\begin{enumerate}[label=\arabic*.,start=3]
    \item 例3 (2001,数一)交换二次积分的积分次序:
    \begin{align*}
        \int_{-1}^0 dy\int_2^{1-y} f(x,y)dx=
    \end{align*}
    
    \begin{solution}
    【详解】
    \end{solution}
    
    \item 例5 交换$I=\int_{-\frac{\pi}{4}}^{\frac{\pi}{2}}d\theta\int_0^{a\cos\theta}f(r,\theta)dr$的积分次序。
    
    \begin{solution}
    【详解】
    \end{solution}
\end{enumerate}

\subsection{二重积分的计算}

\begin{enumerate}[label=\arabic*.,start=6]
    \item 例6 (2011,数一、数二)已知函数$f(x,y)$具有二阶连续偏导数,且$f(1,y)=0$,$f(x,1)=0$,$\iint_D f(x,y)dxdy=a$,其中$D=\{(x,y)|0\leq x\leq 1,0\leq y\leq 1\}$,计算二重积分
    \begin{align*}
        I=\iint_D xyf_{xy}''(x,y)dxdy.
    \end{align*}
    
    \begin{solution}
    【详解】
    \end{solution}
    
    \item 例7 计算$\iint_D\sqrt{|y-x^2|}dxdy$,其中$D=\{(x,y)|-1\leq x\leq 1,0\leq y\leq 2\}$。
    
    \begin{solution}
    【详解】
    \end{solution}
    
    \item 例8 (2018,数二)设平面区域$D$由曲线$\begin{cases}x=t-\sin t \\ y=1-\cos t\end{cases}(0\leq t\leq 2\pi)$与$x$轴围成,计算二重积分$\iint_D(x+2y)dxdy$。
    
    \begin{solution}
    【详解】
    \end{solution}
    
    \item 例9 (2007,数二、数三)设二元函数
    \begin{align*}
        f(x,y)=\begin{cases}
            x^2, & |x|+|y|\leq 1 \\
            \frac{1}{\sqrt{x^2+y^2}}, & 1<|x|+|y|\leq 2
        \end{cases}
    \end{align*}
    计算二重积分$\iint_D f(x,y)dxdy$,其中$D=\{(x,y)||x|+|y|\leq 2\}$。
    
    \begin{solution}
    【详解】
    \end{solution}
    
    \item 例10 (2014,数二、数三)设平面区域$D=\{(x,y)|1\leq x^2+y^2\leq 4,x\geq 0,y\geq 0\}$,计算
    \begin{align*}
        \iint_D\frac{x\sin(\pi\sqrt{x^2+y^2})}{x+y}dxdy.
    \end{align*}
    
    \begin{solution}
    【详解】
    \end{solution}
    
    \item 例11 (2019,数二)已知平面区域$D=\{(x,y)||x|\leq y,(x^2+y^2)^3\leq y^4\}$,计算二重积分
    \begin{align*}
        \iint_D\frac{x+y}{\sqrt{x^2+y^2}}dxdy.
    \end{align*}
    
    \begin{solution}
    【详解】
    \end{solution}
\end{enumerate}

\subsection{其他题型}

\begin{enumerate}[label=\arabic*.,start=13]
    \item 例12 (2010,数二)计算二重积分$I=\iint_D r^2\sin\theta\sqrt{1-r^2\cos 2\theta} drd\theta$,其中(题目描述不完整)
    
    \begin{solution}
    【详解】
    \end{solution}
    
    \item 例13 (2009,数二、数三)计算二重积分$\iint_D(x-y)dxdy$,其中
    \begin{align*}
        D=\{(x,y)|(x-1)^2+(y-1)^2\leq 2,y\geq x\}.
    \end{align*}
    
    \begin{solution}
    【详解】
    \end{solution}
\end{enumerate}

\section{无穷级数}
\subsection{数项级数敛散性的判定}

\begin{enumerate}[label=\arabic*.]
    \item 例1 (2015,数三)下列级数中发散的是
    \begin{align*}
        (A)\sum_{n=1}^{\infty}\frac{n}{3^n} \quad (C)\sum_{n=2}^{\infty}\frac{(-1)^n+1}{\ln n} \quad (D)\sum_{n=1}^{\infty}\frac{n!}{n^n}
    \end{align*}
    
    \begin{solution}
    【详解】
    \end{solution}
    
    \item 例2 (2017,数三)若级数$\sum_{n=1}^{\infty}\left[\sin\frac{1}{n}-k\ln\left(1-\frac{1}{n}\right)\right]$收敛,则$k=$
    \begin{align*}
        (A)\ 1 \quad (B)\ 2 \quad (C)\ -1 \quad (D)\ -2
    \end{align*}
    
    \begin{solution}
    【详解】
    \end{solution}
\end{enumerate}

\subsection{交错级数}

\begin{enumerate}[label=\arabic*.,start=3]
    \item 例3 判定下列级数的敛散性:
    \begin{align*}
        (1)\sum_{n=1}^{\infty}\frac{(-1)^{n-1}}{n-\ln n} \quad (2)\sum_{n=2}^{\infty}\frac{(-1)^n}{\sqrt{n}+(-1)^n}.
    \end{align*}
    
    \begin{solution}
    【详解】
    \end{solution}
\end{enumerate}

\subsection{任意项级数}

\begin{enumerate}[label=\arabic*.,start=4]
    \item 例4 (2002,数一)设$u_n\neq 0(n=1,2,3,\cdots)$,且$\lim_{n\rightarrow\infty}\frac{n}{u_n}=1$,则级数$\sum_{n=1}^{\infty}(-1)^{n+1}\left(\frac{1}{u_n}+\frac{1}{u_{n+1}}\right)$
    \begin{align*}
        (A)\ 发散 \quad (B)\ 绝对收敛 \quad (C)\ 条件收敛 \quad (D)\ 敛散性根据所给条件不能判定
    \end{align*}
    
    \begin{solution}
    【详解】
    \end{solution}
    
    \item 例5 (2019,数三)若$\sum_{n=1}^{\infty}\frac{v_n}{n}$条件收敛,则
    \begin{align*}
        (A)\sum_{n=1}^{\infty} u_n v_n\text{条件收敛} \quad (B)\sum_{n=1}^{\infty} u_n v_n\text{绝对收敛} \\
        (C)\sum_{n=1}^{\infty}\left(u_n+v_n\right)\text{收敛} \quad (D)\sum_{n=1}^{\infty}\left(u_n+v_n\right)\text{发散}
    \end{align*}
    
    \begin{solution}
    【详解】
    \end{solution}
\end{enumerate}

\subsection{幂级数求收敛半径与收敛域}

\begin{enumerate}[label=\arabic*.,start=6]
    \item 例6 (2015,数一)若级数$\sum_{n=1}^{\infty} a_n$条件收敛,则$x=\sqrt{3}$与$x=3$依次为幂级数$\sum_{n=1}^{\infty} n a_n(x-1)^n$的
    \begin{align*}
        (A)\ 收敛点,收敛点 \quad (B)\ 收敛点,发散点 \\
        (C)\ 发散点,收敛点 \quad (D)\ 发散点,发散点
    \end{align*}
    
    \begin{solution}
    【详解】
    \end{solution}
    
    \item 例7 求幂级数$\sum_{n=1}^{\infty}\frac{3n}{2n+1}x^n$的收敛域.
    
    \begin{solution}
    【详解】
    \end{solution}
\end{enumerate}

\subsection{幂级数求和}

\begin{enumerate}[label=\arabic*.,start=8]
    \item 例8 (2005,数一)求幂级数$\sum_{n=1}^{\infty}(-1)^{n-1}\left[1+\frac{1}{n(2n-1)}\right] x^{2n}$的收敛区间与和函数$f(x)$.
    
    \begin{solution}
    【详解】
    \end{solution}
    
    \item 例9 (2012,数一)求幂级数$\sum_{n=0}^{\infty}\frac{4n^2+4n+3}{2n+1} x^{2n}$的收敛域及和函数.
    
    \begin{solution}
    【详解】
    \end{solution}
    
    \item 例10 (2004,数三)设级数$\frac{x^4}{2\cdot 4}+\frac{x^6}{2\cdot 4\cdot 6}+\frac{x^8}{2\cdot 4\cdot 6\cdot 8}+\cdots\quad(-\infty<x<+\infty)$的和函数为$S(x)$。求:
    \begin{enumerate}[label=(\roman*)]
        \item $S(x)$所满足的一阶微分方程;
        \item $S(x)$的表达式.
    \end{enumerate}
    
    \begin{solution}
    【详解】
    \end{solution}
\end{enumerate}

\subsection{幂级数展开}

\begin{enumerate}[label=\arabic*.,start=11]
    \item 例11 (2007,数三)将函数$f(x)=\frac{1}{x^2-3x-4}$展开成$x-1$的幂级数,并指出其收敛区间.
    
    \begin{solution}
    【详解】
    \end{solution}
    
    \item 例12 将函数$f(x)=\ln\frac{x}{x+1}$在$x=1$处展开成幂级数.
    
    \begin{solution}
    【详解】
    \end{solution}
\end{enumerate}

\subsection{无穷级数证明题}

\begin{enumerate}[label=\arabic*.,start=13]
    \item 例13 (2016,数一)已知函数$f(x)$可导,且$f(0)=1$,$0<f'(x)<\frac{1}{2}$。设数列$\{x_n\}$满足$x_{n+1}=f(x_n)(n=1,2,\cdots)$。证明:
    \begin{enumerate}[label=(\roman*)]
        \item 级数$\sum_{n=1}^{\infty}(x_{n+1}-x_n)$绝对收敛;
        \item $\lim_{n\rightarrow\infty} x_n$存在,且$0<\lim_{n\rightarrow\infty} x_n<2$.
    \end{enumerate}
    
    \begin{solution}
    【详解】
    \end{solution}
    
    \item 例14 (2014,数一)设数列$\{a_n\}$,$\{b_n\}$满足$0<a_n<\frac{\pi}{2}$,$0<b_n<\frac{\pi}{2}$,$\cos a_n-a_n=\cos b_n$,且级数$\sum_{n=1}^{\infty} b_n$收敛。
    \begin{enumerate}[label=(\roman*)]
        \item 证明$\lim_{n\rightarrow\infty} a_n=0$;
        \item 证明级数$\sum_{n=1}^{\infty}\frac{a_n}{b_n}$收敛.
    \end{enumerate}
    
    \begin{solution}
    【详解】
    \end{solution}
\end{enumerate}

\subsection{傅里叶级数}

\begin{enumerate}[label=\arabic*.,start=15]
    \item 例15 设函数
    \begin{align*}
    f(x)=\begin{cases}
    e^x, & -\pi\leq x<0 \\
    1, & 0\leq x<\pi
    \end{cases}
    \end{align*}
    则其以$2\pi$为周期的傅里叶级数在$x=\pi$收敛于?,在$x=2\pi$收敛于?.
    \begin{solution}
    【详解】
    由狄利克雷收敛定理知,$f(x)$以$2\pi$为周期的傅里叶级数在$x=\pi$收敛于
    \begin{align*}
    S(\pi)=\frac{f(\pi-0)+f(-\pi+0)}{2}=\frac{1+e^{-\pi}}{2}
    \end{align*}
    在$x=2\pi$收敛于
    \begin{align*}
    S(2\pi)=S(0)=\frac{f(0-0)+f(0+0)}{2}=\frac{1+1}{2}=1
    \end{align*}
    \end{solution}
    
    \item 例16 将$f(x)=1-x^2,0\leq x\leq\pi$,展开成余弦级数,并求级数$\sum_{n=1}^{\infty}\frac{(-1)^{n-1}}{n^2}$的和.
    
    \begin{solution}
    【详解】
    对$f(x)=1-x^2$进行偶延拓,由$f(x)=1-x^2$为偶函数,知$b_n=0$。
    \begin{align*}
    a_0&=\frac{2}{\pi}\int_0^\pi(1-x^2)dx=2\left(1-\frac{\pi^2}{3}\right) \\
    a_n&=\frac{2}{\pi}\int_0^\pi(1-x^2)\cos nx dx=\frac{4(-1)^{n+1}}{n^2} \quad (n=1,2,\cdots)
    \end{align*}
    \begin{align*}
    f(x)=1-x^2=\frac{a_0}{2}+\sum_{n=1}^{\infty}a_n\cos nx=1-\frac{\pi^2}{3}+\sum_{n=1}^{\infty}\frac{4(-1)^{n+1}}{n^2}\cos nx
    \end{align*}
    令$x=0$,代入上式,得
    \begin{align*}
    \sum_{n=1}^{\infty}\frac{(-1)^{n-1}}{n^2}=\frac{\pi^2}{12}
    \end{align*}
    \end{solution}
\end{enumerate}

\section{多元函数积分学}
\subsection{三重积分的计算}

\begin{enumerate}[label=\arabic*.]
    \item 例1 (2013,数一)设直线$L$过$A(1,0,0)$,$B(0,1,1)$两点,将$L$绕$z$轴旋转一周得到曲面$\Sigma$,$\Sigma$与平面$z=0$,$z=2$所围成的立体为$\Omega$.
    \begin{enumerate}
        \item[(I)] 求曲面$\Sigma$的方程;
        \item[(II)] 求$\Omega$的形心坐标.
    \end{enumerate}
    
    \begin{solution}
    【详解】
    \end{solution}
    
    \item 例2 (2019,数一)设$\Omega$是由锥面$x^{2}+(y-z)^{2}=(1-z)^{2}(0\leq z\leq 1)$与平面$z=0$围成的锥体,求$\Omega$的形心坐标.
    
    \begin{solution}
    【详解】
    \end{solution}
\end{enumerate}

\subsection{第一类曲线积分的计算}

\begin{enumerate}[label=\arabic*.,start=3]
    \item 例3 (2018,数一)设$L$为球面$x^2+y^2+z^2=1$与平面$x+y+z=0$的交线,则$\oint_L xy ds=$
    
    \begin{solution}
    【详解】
    \end{solution}
    
    \item 例4 设连续函数$f(x,y)$满足$f(x,y)=(x+3y)^2+\int_L f(x,y) ds$,其中$L$为曲线$y=\sqrt{1-x^2}$,求曲线积分$\int_L f(x,y) ds$.
    
    \begin{solution}
    【详解】
    \end{solution}
\end{enumerate}

\subsection{第二类曲线积分的计算}

\begin{remark}[类型一 平面第二类曲线积分]
\end{remark}

\begin{enumerate}[label=\arabic*.,start=5]
    \item 例5 (2021,数一)设$D\subset \mathbb{R}^2$是有界单连通闭区域,$I(D)=\iint_D(4-x^2-y^2)dxdy$取得最大值的积分域记为$D_1$.
    \begin{enumerate}
        \item[(I)] 求$I(D_1)$的值;
        \item[(II)] 计算$\oint_{\partial D_1}\frac{(xe^{x^2+4y^2}+y)dx+(4ye^{x^2+4y^2}-x)dy}{x^2+4y^2}$,其中$\partial D_1$是$D_1$的正向边界.
    \end{enumerate}
    
    \begin{solution}
    【详解】
    \end{solution}
\end{enumerate}

\begin{remark}[类型二 空间第二类曲线积分]
\end{remark}

\begin{enumerate}[label=\arabic*.,start=6]
    \item 例6 (2011,数一)设$L$是柱面$x^2+y^2=1$与平面$z=x+y$的交线,从$z$轴正向往$z$轴负向看去为逆时针方向,则曲线积分$\oint_L xz dx+xdy+\frac{y^2}{2}dz=$
    
    \begin{solution}
    【详解】
    \end{solution}
\end{enumerate}

\subsection{第一类曲面积分的计算}

\begin{remark}[方法]
\end{remark}

\begin{enumerate}[label=\arabic*.,start=7]
    \item 例7 (2010,数一)设$P$为椭球面$S:x^2+y^2+z^2-yz=1$上的动点,若$S$在点$P$的切平面与$xOy$面垂直,求$P$点的轨迹$C$,并计算曲面积分
    \begin{align*}
    I=\iint_{\Sigma}\frac{(x+\sqrt{3})|y-2z|}{\sqrt{4+y^2+z^2-4yz}}dS,
    \end{align*}
    其中$\Sigma$是椭球面$S$位于曲线$C$上方的部分.
    
    \begin{solution}
    【详解】
    \end{solution}
\end{enumerate}

\subsection{第二类曲面积分的计算}

\begin{remark}[方法]
\end{remark}

\begin{enumerate}[label=\arabic*.,start=8]
    \item 例8 (2009,数一)计算曲面积分
    \begin{align*}
    I=\oint_{\Sigma}\frac{xdydz+ydzdx+zdxdy}{(x^2+y^2+z^2)^{\frac{3}{2}}},
    \end{align*}
    其中$\Sigma$是曲面$2x^2+2y^2+z^2=4$的外侧.
    
    \begin{solution}
    【详解】
    \end{solution}
    
    \item 例9 计算
    \begin{align*}
    \iint_{\Sigma}\frac{axdydz+(z+a)^2dxdy}{(x^2+y^2+z^2)^2},
    \end{align*}
    其中$\Sigma$为下半球面$z=-\sqrt{a^2-x^2-y^2}$的上侧,$a$为大于零的常数.
    
    \begin{solution}
    【详解】
    \end{solution}
    
    \item 例10 (2020,数一)设$\Sigma$为曲面$z=\sqrt{x^2+y^2}(1\leq x^2+y^2\leq 4)$的下侧,$f(x)$为连续函数,计算
    \begin{align*}
    I=\iint_{\Sigma}[xf(xy)+2x-y]dydz+[yf(xy)+2y+x]dzdx+[zf(xy)+z]dxdy.
    \end{align*}
    
    \begin{solution}
    【详解】
    \end{solution}
\end{enumerate}

\ifx\allfiles\undefined
\end{document}
\fi