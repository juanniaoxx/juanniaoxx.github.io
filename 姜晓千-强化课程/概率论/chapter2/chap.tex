\ifx\allfiles\undefined
\documentclass[12pt, a4paper, oneside, UTF8]{ctexbook}
\def\path{../../config}
\usepackage{amsmath}
\usepackage{amsthm}
\usepackage{amssymb}
\usepackage{array}
\usepackage{xcolor}
\usepackage{graphicx}
\usepackage{mathrsfs}
\usepackage{enumitem}
\usepackage{geometry}
\usepackage[colorlinks, linkcolor=black]{hyperref}
\usepackage{stackengine}
\usepackage{yhmath}
\usepackage{extarrows}
\usepackage{tikz}
\usepackage{pgfplots}
\usetikzlibrary{decorations.pathreplacing, positioning}
\usepgfplotslibrary{fillbetween}
% \usepackage{unicode-math}
\usepackage{esint}
\usepackage[most]{tcolorbox}

\usepackage{fancyhdr}
\usepackage[dvipsnames, svgnames]{xcolor}
\usepackage{listings}

\definecolor{mygreen}{rgb}{0,0.6,0}
\definecolor{mygray}{rgb}{0.5,0.5,0.5}
\definecolor{mymauve}{rgb}{0.58,0,0.82}
\definecolor{NavyBlue}{RGB}{0,0,128}
\definecolor{Rhodamine}{RGB}{255,0,255}
\definecolor{PineGreen}{RGB}{0,128,0}

\graphicspath{ {figures/},{../figures/}, {config/}, {../config/} }

\linespread{1.6}

\geometry{
    top=25.4mm, 
    bottom=25.4mm, 
    left=20mm, 
    right=20mm, 
    headheight=2.17cm, 
    headsep=4mm, 
    footskip=12mm
}

\setenumerate[1]{itemsep=5pt,partopsep=0pt,parsep=\parskip,topsep=5pt}
\setitemize[1]{itemsep=5pt,partopsep=0pt,parsep=\parskip,topsep=5pt}
\setdescription{itemsep=5pt,partopsep=0pt,parsep=\parskip,topsep=5pt}

\lstset{
    language=Mathematica,
    basicstyle=\tt,
    breaklines=true,
    keywordstyle=\bfseries\color{NavyBlue}, 
    emphstyle=\bfseries\color{Rhodamine},
    commentstyle=\itshape\color{black!50!white}, 
    stringstyle=\bfseries\color{PineGreen!90!black},
    columns=flexible,
    numbers=left,
    numberstyle=\footnotesize,
    frame=tb,
    breakatwhitespace=false,
} 

\lstset{
    language=TeX, % 设置语言为 TeX
    basicstyle=\ttfamily, % 使用等宽字体
    breaklines=true, % 自动换行
    keywordstyle=\bfseries\color{NavyBlue}, % 关键字样式
    emphstyle=\bfseries\color{Rhodamine}, % 强调样式
    commentstyle=\itshape\color{black!50!white}, % 注释样式
    stringstyle=\bfseries\color{PineGreen!90!black}, % 字符串样式
    columns=flexible, % 列的灵活性
    numbers=left, % 行号在左侧
    numberstyle=\footnotesize, % 行号字体大小
    frame=tb, % 顶部和底部边框
    breakatwhitespace=false % 不在空白处断行
}

% \begin{lstlisting}[language=TeX] ... \end{lstlisting}

% 定理环境设置
\usepackage[strict]{changepage} 
\usepackage{framed}

\definecolor{greenshade}{rgb}{0.90,1,0.92}
\definecolor{redshade}{rgb}{1.00,0.88,0.88}
\definecolor{brownshade}{rgb}{0.99,0.95,0.9}
\definecolor{lilacshade}{rgb}{0.95,0.93,0.98}
\definecolor{orangeshade}{rgb}{1.00,0.88,0.82}
\definecolor{lightblueshade}{rgb}{0.8,0.92,1}
\definecolor{purple}{rgb}{0.81,0.85,1}

\theoremstyle{definition}
\newtheorem{myDefn}{\indent Definition}[section]
\newtheorem{myLemma}{\indent Lemma}[section]
\newtheorem{myThm}[myLemma]{\indent Theorem}
\newtheorem{myCorollary}[myLemma]{\indent Corollary}
\newtheorem{myCriterion}[myLemma]{\indent Criterion}
\newtheorem*{myRemark}{\indent Remark}
\newtheorem{myProposition}{\indent Proposition}[section]

\newenvironment{formal}[2][]{%
	\def\FrameCommand{%
		\hspace{1pt}%
		{\color{#1}\vrule width 2pt}%
		{\color{#2}\vrule width 4pt}%
		\colorbox{#2}%
	}%
	\MakeFramed{\advance\hsize-\width\FrameRestore}%
	\noindent\hspace{-4.55pt}%
	\begin{adjustwidth}{}{7pt}\vspace{2pt}\vspace{2pt}}{%
		\vspace{2pt}\end{adjustwidth}\endMakeFramed%
}

\newenvironment{definition}{\vspace{-\baselineskip * 2 / 3}%
	\begin{formal}[Green]{greenshade}\vspace{-\baselineskip * 4 / 5}\begin{myDefn}}
	{\end{myDefn}\end{formal}\vspace{-\baselineskip * 2 / 3}}

\newenvironment{theorem}{\vspace{-\baselineskip * 2 / 3}%
	\begin{formal}[LightSkyBlue]{lightblueshade}\vspace{-\baselineskip * 4 / 5}\begin{myThm}}%
	{\end{myThm}\end{formal}\vspace{-\baselineskip * 2 / 3}}

\newenvironment{lemma}{\vspace{-\baselineskip * 2 / 3}%
	\begin{formal}[Plum]{lilacshade}\vspace{-\baselineskip * 4 / 5}\begin{myLemma}}%
	{\end{myLemma}\end{formal}\vspace{-\baselineskip * 2 / 3}}

\newenvironment{corollary}{\vspace{-\baselineskip * 2 / 3}%
	\begin{formal}[BurlyWood]{brownshade}\vspace{-\baselineskip * 4 / 5}\begin{myCorollary}}%
	{\end{myCorollary}\end{formal}\vspace{-\baselineskip * 2 / 3}}

\newenvironment{criterion}{\vspace{-\baselineskip * 2 / 3}%
	\begin{formal}[DarkOrange]{orangeshade}\vspace{-\baselineskip * 4 / 5}\begin{myCriterion}}%
	{\end{myCriterion}\end{formal}\vspace{-\baselineskip * 2 / 3}}
	

\newenvironment{remark}{\vspace{-\baselineskip * 2 / 3}%
	\begin{formal}[LightCoral]{redshade}\vspace{-\baselineskip * 4 / 5}\begin{myRemark}}%
	{\end{myRemark}\end{formal}\vspace{-\baselineskip * 2 / 3}}

\newenvironment{proposition}{\vspace{-\baselineskip * 2 / 3}%
	\begin{formal}[RoyalPurple]{purple}\vspace{-\baselineskip * 4 / 5}\begin{myProposition}}%
	{\end{myProposition}\end{formal}\vspace{-\baselineskip * 2 / 3}}


\newtheorem{example}{\indent \color{SeaGreen}{Example}}[section]
\renewcommand{\proofname}{\indent\textbf{\textcolor{TealBlue}{Proof}}}
\newenvironment{solution}{\begin{proof}[\indent\textbf{\textcolor{TealBlue}{Solution}}]}{\end{proof}}

% 自定义命令的文件

\def\d{\mathrm{d}}
\def\R{\mathbb{R}}
%\newcommand{\bs}[1]{\boldsymbol{#1}}
%\newcommand{\ora}[1]{\overrightarrow{#1}}
\newcommand{\myspace}[1]{\par\vspace{#1\baselineskip}}
\newcommand{\xrowht}[2][0]{\addstackgap[.5\dimexpr#2\relax]{\vphantom{#1}}}
\newenvironment{mycases}[1][1]{\linespread{#1} \selectfont \begin{cases}}{\end{cases}}
\newenvironment{myvmatrix}[1][1]{\linespread{#1} \selectfont \begin{vmatrix}}{\end{vmatrix}}
\newcommand{\tabincell}[2]{\begin{tabular}{@{}#1@{}}#2\end{tabular}}
\newcommand{\pll}{\kern 0.56em/\kern -0.8em /\kern 0.56em}
\newcommand{\dive}[1][F]{\mathrm{div}\;\boldsymbol{#1}}
\newcommand{\rotn}[1][A]{\mathrm{rot}\;\boldsymbol{#1}}

% 修改参数改变封面样式,0 默认原始封面、内置其他1、2、3种封面样式
\def\myIndex{0}


\ifnum\myIndex>0
    \input{\path/cover_package_\myIndex} 
\fi

\def\myTitle{姜晓千 2023年强化班笔记}
\def\myAuthor{Weary Bird}
\def\myDateCover{\today}
\def\myDateForeword{\today}
\def\myForeword{相见欢·林花谢了春红}
\def\myForewordText{
    林花谢了春红,太匆匆。
    无奈朝来寒雨晚来风。
    胭脂泪,相留醉,几时重。
    自是人生长恨水长东。
}
\def\mySubheading{数学笔记}


\begin{document}
% \input{\path/cover_text_\myIndex.tex}

\newpage
\thispagestyle{empty}
\begin{center}
    \Huge\textbf{\myForeword}
\end{center}
\myForewordText
\begin{flushright}
    \begin{tabular}{c}
        \myDateForeword
    \end{tabular}
\end{flushright}

\newpage
\pagestyle{plain}
\setcounter{page}{1}
\pagenumbering{Roman}
\tableofcontents

\newpage
\pagenumbering{arabic}
% \setcounter{chapter}{-1}
\setcounter{page}{1}

\pagestyle{fancy}
\fancyfoot[C]{\thepage}
\renewcommand{\headrulewidth}{0.4pt}
\renewcommand{\footrulewidth}{0pt}








\else
\fi

\chapter{一维随机变量}

\section{分布函数的判定与计算}

\begin{enumerate}[label=\arabic*.]
    % 例题 2.1
    \item 设随机变量$X$的分布函数为$F(x)$,$a,b$为任意常数,则下列一定不是分布函数的是
    \begin{align*}
        (A)\ F(ax+b) \quad (B)\ F(x^2+b) \quad (C)\ F(x^3+b) \quad (D)\ 1-F(-x)
    \end{align*}
    
    \begin{solution}
    【详解】
    \end{solution}
    
    \item 设随机变量$X$的概率密度为
    \begin{align*}
        f(x)=\begin{cases}
            1-|x|, & |x|<1 \\
            0, & \text{其他}
        \end{cases}
    \end{align*}
    则$X$的分布函数$F(x)=$?,$P\{-2<X<\frac{1}{4}\}=$?.
    
    \begin{solution}
    【详解】
    \end{solution}
\end{enumerate}

\section{概率密度的判定与计算}

\begin{enumerate}[label=\arabic*.,start=3]
    \item 设随机变量$X$的概率密度为$f(x)$,则下列必为概率密度的是
    \begin{align*}
        (A)\ f(-x+1) \quad (B)\ f(2x-1) \quad (C)\ f(-2x+1) \quad (D)\ f\left(\frac{1}{2}x-1\right)
    \end{align*}
    
    \begin{solution}
    【详解】
    \end{solution}
    
    \item (2011,数一、三)设$F_1(x),F_2(x)$为分布函数,对应的概率密度$f_1(x),f_2(x)$为连续函数,则下列必为概率密度的是
    \begin{align*}
        (A)\ f_1(x)f_2(x) \quad (B)\ 2f_2(x)F_1(x) \quad (C)\ f_1(x)F_2(x) \quad (D)\ f_1(x)F_2(x)+f_2(x)F_1(x)
    \end{align*}
    
    \begin{solution}
    【详解】
    \end{solution}
    
    \item (2000,三)设随机变量$X$的概率密度为
    \begin{align*}
        f(x)=\begin{cases}
            \frac{1}{3}, & x\in[0,1] \\
            \frac{2}{9}, & x\in[3,6] \\
            0, & \text{其他}
        \end{cases}
    \end{align*}
    若$P\{X\geq k\}=\frac{2}{3}$,则$k$的取值范围是?.
    
    \begin{solution}
    【详解】
    \end{solution}
\end{enumerate}

\section{关于八大分布}

\begin{enumerate}[label=\arabic*.,start=6]
    \item 设随机变量$X$的概率分布为$P\{X=k\}=C\frac{\lambda^k}{k!}$,$k=1,2,\cdots$,则$C=$?.
    
    \begin{solution}
    【详解】
    \end{solution}
    
    \item 设随机变量$X$的概率密度为$f(x)=Ae^{-\frac{x^2}{2}+Bx}$,且$EX=DX$,则$A=$?,$B=$?.
    
    \begin{solution}
    【详解】
    \end{solution}
    
    \item (2004,数一、三)设随机变量$X\sim N(0,1)$,对给定的$\alpha(0<\alpha<1)$,数$u_\alpha$满足$P\{X>u_\alpha\}=\alpha$。若$P\{|X|<x\}=\alpha$,则$x$等于
    \begin{align*}
        (A)\ u_{\frac{\alpha}{2}} \quad (B)\ u_{1-\frac{\alpha}{2}} \quad (C)\ u_{\frac{1-\alpha}{2}} \quad (D)\ u_{1-\alpha}
    \end{align*}
    
    \begin{solution}
    【详解】
    \end{solution}
    
    \item 设随机变量$X\sim N(2,\sigma^2)$,且$P\{2<X<4\}=0.3$,则$P\{X<0\}=$?.
    
    \begin{solution}
    【详解】
    \end{solution}
    
    \item  设随机变量$X\sim N(\mu,\sigma^2)(\mu<0)$,$F(x)$为其分布函数,$a$为任意常数,则
    \begin{align*}
        (A)\ F(a)+F(-a)>1 \quad (B)\ F(a)+F(-a)=1 \\
        (C)\ F(a)+F(-a)<1 \quad (D)\ F(\mu+a)+F(\mu-a)=\frac{1}{2}
    \end{align*}
    
    \begin{solution}
    【详解】
    \end{solution}
    
    \item  设随机变量$X$与$Y$相互独立,均服从参数为1的指数分布,则$P\{1<\max\{X,Y\}<2\}=$?.
    
    \begin{solution}
    【详解】
    \end{solution}
    
    \item  设随机变量$X$与$Y$相互独立,均服从区间$[0,3]$上的均匀分布,则$P\{1<\min\{X,Y\}<2\}=$?.
    
    \begin{solution}
    【详解】
    \end{solution}
    
    \item  (2013,数一)设随机变量$Y\sim E(1)$,$a>0$,则$P\{Y\leq a+1|Y>a\}=$?.
    
    \begin{solution}
    【详解】
    \end{solution}
    
    \item  设随机变量$X\sim G(p)$,$m,n$为正整数,则$P\{X>m+n|X>m\}$
    \begin{align*}
        (A)\ 与m无关,与n有关,且随n的增大而减少 \\
        (B)\ 与m无关,与n有关,且随n的增大而增大 \\
        (C)\ 与n无关,与m有关,且随m的增大而减少 \\
        (D)\ 与n无关,与m有关,且随m的增大而增大
    \end{align*}
    
    \begin{solution}
    【详解】
    \end{solution}
\end{enumerate}

\section{求一维连续型随机变量函数的分布}

\begin{enumerate}[label=\arabic*.,start=15]
    \item  设随机变量$X\sim E(\lambda)$,则$Y=\min\{X,2\}$的分布函数
    \begin{align*}
        (A)\ 为连续函数 \quad (B)\ 为阶梯函数 \\
        (C)\ 至少有两个间断点 \quad (D)\ 恰好有一个间断点
    \end{align*}
    
    \begin{solution}
    【详解】
    \end{solution}
    
    \item  (2013,数一)设随机变量$X$的概率密度为
    \begin{align*}
        f(x)=\begin{cases}
            \frac{2x}{a^2}, & 0<x<a \\
            0, & \text{其他}
        \end{cases}
    \end{align*}
    $Y=\begin{cases}
        \frac{1}{3}X, & X\leq 1 \\
        X, & 1<X<2 \\
        1, & X\geq 2
    \end{cases}$
    \begin{enumerate}
        \item 求$Y$的分布函数;
        \item 求$P\{X\leq Y\}$.
    \end{enumerate}
    
    \begin{solution}
    【详解】
    \end{solution}
    
    \item  (2021,数一、三)在区间$(0,2)$上随机取一点,将该区间分成两段,较短一段的长度记为$X$,较长一段的长度记为$Y$。
    \begin{enumerate}
        \item 求$X$的概率密度;
        \item 求$Z=\frac{Y}{X}$的概率密度;
        \item 求$E\left(\frac{Y}{X}\right)$.
    \end{enumerate}
    
    \begin{solution}
    【详解】
    \end{solution}
\end{enumerate}

\ifx\allfiles\undefined
\end{document}
\fi