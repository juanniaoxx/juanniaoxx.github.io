\ifx\allfiles\undefined
\documentclass[12pt, a4paper, oneside, UTF8]{ctexbook}
\def\path{../config}
\input{../config/_config}
\begin{document}
% \input{../config/cover}
\else
\fi

\chapter{线性代数部分}

\section{行列式}

\subsection{数字行列式的计算}

\begin{enumerate}[label=\arabic*.]
    \item 设
    \begin{align*}
    f(x)=\left|\begin{array}{lllll}
    x-2 & x-1 & x-2 & x-3 \\
    2 x-2 & 2 x-1 & 2 x-2 & 2 x-3 \\
    3 x-3 & 3 x-2 & 4 x-5 & 3 x-5 \\
    4 x & 4 x-3 & 5 x-7 & 4 x-3
    \end{array}\right|
    \end{align*}
    则方程 $f(x)=0$ 根的个数为?    
    \begin{solution}
    【详解】
    \end{solution}
    
    \item 利用范德蒙行列式计算 
    \begin{align*}
    \left|\begin{array}{lll}
    b & b^2 & a c \\
    c & c^2 & a b
    \end{array}\right|=
    \end{align*}
    
    \begin{solution}
    【详解】
    \end{solution}
    
    \item 设 $x_1, x_2, x_3, x_4$,
    
    \begin{solution}
    【详解】
    \end{solution}
    
    \item 计算三对角线行列式
    \begin{align*}
    D_{n}= 
    \begin{vmatrix}
    \alpha+\beta & \alpha & 0 & \cdots & 0 & 0 \\
    \beta & \alpha+\beta & \alpha & \cdots & 0 & 0 \\
    0 & \beta & \alpha+\beta & \cdots & 0 & 0 \\
    \vdots & \vdots & \vdots & \ddots & \vdots & \vdots \\
    0 & 0 & 0 & \cdots & \alpha+\beta & \alpha \\
    0 & 0 & 0 & \cdots & \beta & \alpha+\beta
    \end{vmatrix}
    \end{align*}
    
    \begin{solution}
    【详解】
    \end{solution}
\end{enumerate}

\subsection{代数余子式求和}

\begin{enumerate}[label=\arabic*.,start=4]
    \item 已知 
    \begin{align*}
    |A|=\left|\begin{array}{lllll}
    1 & 2 & 3 & 4 & 5 \\
    2 & 2 & 2 & 1 & 1 \\
    3 & 1 & 2 & 4 & 5 \\
    1 & 1 & 1 & 2 & 2 \\
    4 & 3 & 1 & 5 & 0
    \end{array}\right|=27
    \end{align*}
    则 $A_{41}+A_{42}+A_{43}=$ \underline{\hspace{3cm}}, $A_{44}+A_{45}=$ \underline{\hspace{3cm}}
    
    \begin{solution}
    【详解】
    \end{solution}
    
    \item 设 
    \begin{align*}
    A=\begin{pmatrix}
    0 & 0 & \cdots & 0 & 1 \\
    0 & 0 & \cdots & 2 & 0 \\
    \vdots & \vdots & \ddots & \vdots & \vdots \\
    0 & n-1 & \cdots & 0 & 0 \\
    n & 0 & \cdots & 0 & 0
    \end{pmatrix}
    \end{align*}
    则 $|A|$ 的所有代数余子式的和为\underline{\hspace{3cm}}
    
    \begin{solution}
    【详解】
    \end{solution}
\end{enumerate}

\subsection{抽象行列式的计算}

\begin{enumerate}[label=\arabic*.,start=6]
    \item (2005,数一、二) 设 $\alpha_1,\alpha_2,\alpha_3$ 均为 3维列向量, $A=(\alpha_1,\alpha_2,\alpha_3)$,
    $B=(\alpha_1+\alpha_2+\alpha_3,\alpha_1+2\alpha_2+4\alpha_3,\alpha_1+3\alpha_2+9\alpha_3)$.若 $|A|=1$,则 $|B|=$ \underline{\hspace{3cm}}
    
    \begin{solution}
    【详解】
    \end{solution}
    
    \item 设 A为 n阶矩阵, $\alpha,\beta$ 为 n维列向量.若 $|A|=a$,
    $\left|\begin{array}{ll}A & \alpha \\ \beta^{T} & b\end{array}\right|=0$,则
    $\left|\begin{array}{ll}A & \alpha \\ \beta^{T} & c\end{array}\right|=$ \underline{\hspace{3cm}}
    
    \begin{solution}
    【详解】
    \end{solution}
    
    \item 设 A为 2阶矩阵, $B=\begin{pmatrix}2 & 4 \\ 2 & 2\end{pmatrix}A^2$.若 $|A|=-1$,则 $|B|=$ \underline{\hspace{3cm}}
    
    \begin{solution}
    【详解】
    \end{solution}
    
    \item  设 n阶矩阵 A满足 $A^2=A$, $A\neq E$,证明 $|A|=0$
    
    \begin{solution}
    【详解】
    \end{solution}
\end{enumerate}

\section{矩阵}
\subsection{求高次幂}

\begin{enumerate}[label=\arabic*.]
    \item 设 $ A = \sqrt{a} $,$ B $ 为3阶矩阵,满足 $ BA = O $,且 $ r(B) > 1 $,则 $ A^n = 0 $。
    
    \begin{solution}
    【详解】
    \end{solution}
    
    \item 设 
    \begin{align*}
    A = \begin{pmatrix}
    2 & 0 & 0 \\
    0 & -3 & 2 \\
    0 & 4 & 1
    \end{pmatrix}
    \end{align*}
    则 $ A^n = $ \underline{\hspace{3cm}}。
    
    \begin{solution}
    【详解】
    \end{solution}
    
    \item 设 
    \begin{align*}
    A = \begin{pmatrix}
    -1 & 2 & -1 \\
    -1 & 2 & -1 \\
    -3 & 6 & -3
    \end{pmatrix}
    \end{align*}
    $ P $ 为3阶可逆矩阵,$ B = P^{-1}AP $,则 $ (B + E)^{100} = $ \underline{\hspace{3cm}}。
    
    \begin{solution}
    【详解】
    \end{solution}
\end{enumerate}

\subsection{逆的判定与计算}

\begin{enumerate}[label=\arabic*.,start=3]
    \item 设 $ n $ 阶矩阵 $ A $ 满足 $ A^2 = 2A $,则下列结论不正确的是:
    
    \begin{solution}
    【详解】
    \end{solution}
    
    \item 设 $ A, B $ 为 $ n $ 阶矩阵,$ a, b $ 为非零常数。证明:
    \begin{enumerate}
        \item 若 $ AB = aA + bB $,则 $ AB = BA $;
        \item 若 $ A^2 + aAB = E $,则 $ AB = BA $。
    \end{enumerate}
    
    \begin{solution}
    【详解】
    \end{solution}
    
    \item 设 
    \begin{align*}
    A = \begin{pmatrix}
    a & 1 & 0 \\
    1 & a & -1 \\
    0 & 1 & a
    \end{pmatrix}
    \end{align*}
    满足 $ A^3 = O $。
    \begin{enumerate}
        \item 求 $ a $ 的值;
        \item 若矩阵 $ X $ 满足 $ X - XA^2 - AX + AXA^2 = E $,求 $ X $。
    \end{enumerate}
    
    \begin{solution}
    【详解】
    \end{solution}
\end{enumerate}

\subsection{秩的计算与证明}

\begin{enumerate}[label=\arabic*.,start=6]
    \item (2018, 数一、二、三) 设 $ A, B $ 为 $ n $ 阶矩阵,$ (XY) $ 表示分块矩阵,则:
    \begin{enumerate}
        \item $ r(A A B) = r(A) $
        \item $ r(A B A) = r(A) $
        \item $ r(A B) = \max\{r(A), r(B)\} $
        \item $ r(A B) = r(A^T B^T) $
    \end{enumerate}
    
    \begin{solution}
    【详解】
    \end{solution}
    
    \item (1) 若 $ A^2 = A $,则 $ r(A) + r(A - E) = n $。
    \item (II) 若 $ A^2 = E $,则 $ r(A + E) + r(A - E) = n $。
    
    \begin{solution}
    【详解】
    \end{solution}
\end{enumerate}

\subsection{关于伴随矩阵}

\begin{enumerate}[label=\arabic*.,start=8]
    \item 设 $ n $ 阶矩阵 $ A $ 的各列元素之和均为 2,且 $ |A| = 6 $,则 $ A^* $ 的各列元素之和均为:
    \begin{enumerate}
        \item (A) 2
        \item (B) 1
        \item (C) 3
        \item (D) 6
    \end{enumerate}
    
    \begin{solution}
    【详解】
    \end{solution}
    
    \item 设 $ A = (a_{ij}) $ 为 $ n(n \geq 3) $ 阶非零矩阵,$ A_{ij} $ 为 $ a_{ij} $ 的代数余子式,证明:
    \begin{enumerate}
        \item $ a_{ij} = A_{ij}(i, j = 1, 2, \cdots, n) \Leftrightarrow A^* = A^T \Leftrightarrow AA^T = E $ 且 $ |A| = 1 $;
        \item $ a_{ij} = -A_{ij}(i, j = 1, 2, \cdots, n) \Leftrightarrow A^* = -A^T \Leftrightarrow AA^T = E $ 且 $ |A| = -1 $。
    \end{enumerate}
    
    \begin{solution}
    【详解】
    \end{solution}
\end{enumerate}

\subsection{初等变换与初等矩阵}

\begin{enumerate}[label=\arabic*.,start=10]
    \item (2005, 数一、二) 设 $ A $ 为 $ n(n \geq 2) $ 阶可逆矩阵,交换 $ A $ 的第 1 行与第 2 行得到矩阵 $ B $,则:
    \begin{enumerate}
        \item (A) 交换 $ A^* $ 的第 1 列与第 2 列,得 $ B^* $
        \item (C) 交换 $ A^* $ 的第 1 列与第 2 列,得 $ -B^* $
        \item (D) 交换 $ A $ 的第 1 行与第 2 行,得 $ -B^* $
    \end{enumerate}
    
    \begin{solution}
    【详解】
    \end{solution}
    
    \item 设 
    \begin{align*}
    A = \begin{pmatrix}
    2 & 3 & 0 & 0 \\
    1 & 2 & 0 & 0 \\
    0 & 0 & 0 & 0 \\
    0 & 0 & 0 & 0
    \end{pmatrix}, \quad
    P = \begin{pmatrix}
    0 & 1 & 0 & 0 \\
    1 & 0 & 0 & 0 \\
    0 & 0 & 1 & 0 \\
    0 & 0 & 0 & 1
    \end{pmatrix}, \quad
    Q = \begin{pmatrix}
    1 & 1 & 0 \\
    0 & 1 & 0 \\
    1 & 0 & 0
    \end{pmatrix}
    \end{align*}
    则 $ (P^{-1})^{2023} A (Q^T)^{2022} = $ \underline{\hspace{3cm}}。
    
    \begin{solution}
    【详解】
    \end{solution}
\end{enumerate}

\section{向量}

\subsection{线性表示的判定与计算}

\begin{enumerate}[label=\arabic*.]
    \item 设向量组 $\alpha, \beta, \gamma$ 与数 $k, l, m$ 满足 $k\alpha + l\beta + m\gamma = 0$ ($km \neq 0$),则
    \begin{enumerate}
        \item (A) $\alpha, \beta$ 与 $\alpha, \gamma$ 等价
        \item (B) $\alpha, \beta$ 与 $\beta, \gamma$ 等价
        \item (C) $\alpha, \gamma$ 与 $\beta, \gamma$ 等价
        \item (D) $\alpha$ 与 $\gamma$ 等价
    \end{enumerate}
    
    \begin{solution}
    【详解】
    \end{solution}
    
    \item (2004, 数三) 设 $\alpha_1 = (1,2,0)^T$, $\alpha_2 = (1, a+2, -3a)^T$, $\alpha_3 = (-1, -b-2, a+2b)^T$,
    $\beta = (1,3,-3)^T$。当 $a, b$ 为何值时,
    \begin{enumerate}
        \item (II) $\beta$ 可由 $\alpha_1, \alpha_2, \alpha_3$ 唯一地线性表示,并求出表示式;
        \item (III) $\beta$ 可由 $\alpha_1, \alpha_2, \alpha_3$ 线性表示,但表示式不唯一,并求出表示式。
    \end{enumerate}
    
    \begin{solution}
    【详解】
    \end{solution}
    
    \item (2019, 数二、三) 设向量组 (I) $\alpha_1 = (1,1,4)^T$, $\alpha_2 = (1,0,4)^T$, $\alpha_3 = (1,2, a^2+3)^T$;
    向量组 (II) $\beta_1 = (1,1, a+3)^T$, $\beta_2 = (0,2,1-a)^T$, $\beta_3 = (1,3, a^2+3)^T$。若向量组 (I) 与 (II) 等价,求 $a$ 的值,
    并将 $\beta_3$ 由 $\alpha_1, \alpha_2, \alpha_3$ 线性表示。
    
    \begin{solution}
    【详解】
    \end{solution}
\end{enumerate}

\subsection{线性相关与线性无关的判定}

\begin{enumerate}[label=\arabic*.,start=3]
    \item (2014, 数一、二、三) 设 $\alpha_1, \alpha_2, \alpha_3$ 均为 3 维列向量,则对任意常数 $k, l$,$\alpha_1 + k\alpha_3$, $\alpha_2 + l\alpha_3$ 线性无关是 $\alpha_1, \alpha_2, \alpha_3$ 线性无关的
    \begin{enumerate}
        \item (A) 必要非充分条件
        \item (B) 充分非必要条件
        \item (C) 充分必要条件
        \item (D) 既非充分又非必要条件
    \end{enumerate}
    
    \begin{solution}
    【详解】
    \end{solution}
    
    \item 设 $A$ 为 $n$ 阶矩阵,$\alpha_1, \alpha_2, \alpha_3$ 均为 $n$ 维列向量,满足 $A^2\alpha_1 = A\alpha_1 \neq 0$, $A^2\alpha_2 = \alpha_1 + A\alpha_2$,
    $A^2\alpha_3 = \alpha_2 + A\alpha_3$,证明 $\alpha_1, \alpha_2, \alpha_3$ 线性无关。
    
    \begin{solution}
    【详解】
    \end{solution}
    
    \item 设 4 维列向量 $\alpha_1, \alpha_2, \alpha_3$ 线性无关,与 4 维列向量 $\beta_1, \beta_2$ 两两正交,证明 $\beta_1, \beta_2$ 线性相关。
    
    \begin{solution}
    【详解】
    \end{solution}
\end{enumerate}

\subsection{极大线性无关组的判定与计算}

\begin{enumerate}[label=\arabic*.,start=6]
    \item 设 $\alpha_1 = (1,1,1,3)^T$, $\alpha_2 = (-1,-3,5,1)^T$, $\alpha_3 = (3,2,-1, a+2)^T$, $\alpha_4 = (-2,-6,10, a)^T$。
    \begin{enumerate}
        \item (I) 当 $a$ 为何值时,该向量组线性相关,并求其一个极大线性无关组;
        \item (II) 当 $a$ 为何值时,该向量组线性无关,并将 $\alpha = (4,1,6,10)^T$ 由其线性表示。
    \end{enumerate}
    
    \begin{solution}
    【详解】
    \end{solution}
    
    \item 证明:
    \begin{enumerate}
        \item (I) 设 $A, B$ 为 $m \times n$ 矩阵,则 $r(A+B) \leq r(A) + r(B)$;
        \item (II) 设 $A$ 为 $m \times n$ 矩阵,$B$ 为 $n \times s$ 矩阵,则 $r(AB) \leq \min\{r(A), r(B)\}$。
    \end{enumerate}
    
    \begin{solution}
    【详解】
    \end{solution}
\end{enumerate}

\subsection{向量空间(数一专题)}

\begin{enumerate}[label=\arabic*.,start=8]
    \item (2015, 数一) 设向量组 $\alpha_1, \alpha_2, \alpha_3$ 为 $R^3$ 的一个基,$\beta_1 = 2\alpha_1 + 2k\alpha_3$, $\beta_2 = 2\alpha_2$,
    $\beta_3 = \alpha_1 + (k+1)\alpha_3$。
    \begin{enumerate}
        \item (I) 证明向量组 $\beta_1, \beta_2, \beta_3$ 为 $R^3$ 的一个基:
        \item (II) 当 $k$ 为何值时,存在非零向量 $\xi$ 在基 $\alpha_1, \alpha_2, \alpha_3$ 与基 $\beta_1, \beta_2, \beta_3$ 下的坐标相同,并求所有的 $\xi$。
    \end{enumerate}
    
    \begin{solution}
    【详解】
    \end{solution}
\end{enumerate}

\section{线性方程组}

\subsection{解的判定}

\begin{enumerate}[label=\arabic*.]
    \item (2001,数三) 设 $A$ 为 $n$ 阶矩阵, $\alpha$ 为 $n$ 维列向量, 且 $\begin{pmatrix} A & \alpha \\ \alpha^T & 0 \end{pmatrix} = r(A)$,则线性方程组
    \begin{enumerate}
        \item (A) $Ax = \alpha$ 有无穷多解
        \item (B) $Ax = \alpha$ 有唯一解
        \item (C) $\begin{pmatrix} A & \alpha \\ \alpha^T & 0 \end{pmatrix} \begin{pmatrix} x \\ y \end{pmatrix} = 0$ 只有零解
        \item (D) $\begin{pmatrix} A & \alpha \\ \alpha^T & 0 \end{pmatrix} \begin{pmatrix} x \\ y \end{pmatrix} = 0$ 有非零解
    \end{enumerate}
    
    \begin{solution}
    【详解】
    \end{solution}
    
    \item 设 $A$ 为 $m \times n$ 阶矩阵, 且 $r(A) = m < n$,则下列结论不正确的是
    \begin{enumerate}
        \item (A) 线性方程组 $A^T x = 0$ 只有零解
        \item (B) 线性方程组 $A^T A x = 0$ 有非零解
        \item (C) $\forall b$,线性方程组 $A^T x = b$ 有唯一解
        \item (D) $\forall b$,线性方程组 $A x = b$ 有无穷多解
    \end{enumerate}
    
    \begin{solution}
    【详解】
    \end{solution}
\end{enumerate}

\subsection{求齐次线性方程组的基础解系与通解}

\begin{enumerate}[label=\arabic*.,start=2]
    \item (2011, 数一,二) 设 $A = (\alpha_1, \alpha_2, \alpha_3, \alpha_4)$ 为 4 阶矩阵, $(1,0,1,0)^T$ 为线性方程组 $Ax = 0$ 的基础解系,则 $A^* x = 0$ 的基础解系可为
    \begin{enumerate}
        \item (A) $\alpha_1, \alpha_2$
        \item (B) $\alpha_1, \alpha_3$
        \item (C) $\alpha_1, \alpha_2, \alpha_3$
        \item (D) $\alpha_2, \alpha_3, \alpha_4$
    \end{enumerate}
    
    \begin{solution}
    【详解】
    \end{solution}
    
    \item (2005, 数一、二) 设 3 阶矩阵 $A$ 的第 1 行为 $(a, b, c)$, $a, b, c$ 不全为零, $B = \begin{pmatrix} 2 & 4 & 6 \\ 3 & 6 & k \end{pmatrix}$ 满足 $AB = O$,求线性方程组 $Ax = 0$ 的通解。
    
    \begin{solution}
    【详解】
    \end{solution}
    
    \item (2002, 数三) 设线性方程组
    \begin{align*}
    a x_1 + b x_2 + b x_3 + \cdots + b x_n &= 0 \\
    b x_1 + a x_2 + b x_3 + \cdots + b x_n &= 0 \\
    \vdots \\
    b x_1 + b x_2 + b x_3 + \cdots + a x_n &= 0
    \end{align*}
    其中 $a \neq 0, b \neq 0, n \geq 2$。当 $a, b$ 为何值时,方程组只有零解、有非零解,当方程组有非零解时,求其通解。
    
    \begin{solution}
    【详解】
    \end{solution}
\end{enumerate}

\subsection{求非齐次线性方程组的通解}

\begin{enumerate}[label=\arabic*.,start=5]
    \item 设 $A$ 为 4 阶矩阵, $k$ 为任意常数, $\eta_1, \eta_2, \eta_3$ 为非齐次线性方程组 $Ax = b$ 的三个解, 满足
    \begin{align*}
    \eta_1 + \eta_2 &= \begin{pmatrix} 1 \\ 2 \\ 3 \\ 4 \end{pmatrix}, \quad \eta_2 + 2\eta_3 = \begin{pmatrix} 2 \\ 3 \\ 4 \\ 5 \end{pmatrix}.
    \end{align*}
    
    \begin{solution}
    【详解】
    \end{solution}
    
    \item (2017, 数一、三、三) 设 3 阶矩阵 $A = (\alpha_1', \alpha_2', \alpha_3')$ 有三个不同的特征值, 其中 $\alpha_3 = \alpha_1 + 2\alpha_2$。
    \begin{enumerate}
        \item (I) 证明 $r(A) = 2$;
        \item (II) 若 $\beta = \alpha_1 + \alpha_2 + \alpha_3$,求线性方程组 $Ax = \beta$ 的通解。
    \end{enumerate}
    
    \begin{solution}
    【详解】
    \end{solution}
    
    \item (I) 求 $\lambda, a$ 的值;
    \item (II) 求方程组 $Ax = b$ 的通解。
    
    \begin{solution}
    【详解】
    \end{solution}
    
    \item (I) $\eta$ 为非齐次线性方程组 $Ax = b$ 的特解, 证明:
    \begin{enumerate}
        \item (II) $\eta, \eta + \xi_1, \eta + \xi_2, \cdots, \eta + \xi_{n-r}$ 线性无关;
        \item (III) $\eta, \eta + \xi_1, \eta + \xi_2, \cdots, \eta + \xi_{n-r}$ 为 $Ax = b$ 所有解的极大线性无关组。
    \end{enumerate}
    
    \begin{solution}
    【详解】
    \end{solution}
\end{enumerate}

\subsection{解矩阵方程}

\begin{enumerate}[label=\arabic*.,start=9]
    \item 矩阵方程解的判定
    \begin{align*}
    AX = B \text{ 无解 } \Leftrightarrow r(A) < r(A|B) \\
    AX = B \text{ 有唯一解 } \Leftrightarrow r(A) = r(A|B) = n \\
    AX = B \text{ 有无穷多解 } \Leftrightarrow r(A) = r(A|B) < n
    \end{align*}
    
    \item 矩阵方程的求法
    对 $(A|B)$ 作初等行变换,化为行最简形矩阵,得矩阵 $X$。
    
    \item (例 4.10) 设 
    \begin{align*}
    A = \begin{pmatrix}
    1 & -2 & 3 & -4 \\
    0 & 1 & -1 & 1 \\
    1 & 2 & 0 & -3
    \end{pmatrix}
    \end{align*}
    矩阵 $X$ 满足 $AX + E = A^{2022} + 2X$,求矩阵 $X$。
    
    \begin{solution}
    【详解】
    \end{solution}
    
    \item (例 4.11) (2014, 数一、二、三) 设 
    \begin{align*}
    A = \begin{pmatrix}
    1 & -2 & 3 & -4 \\
    0 & 1 & -1 & 1 \\
    1 & 2 & 0 & -3
    \end{pmatrix}
    \end{align*}
    \begin{enumerate}
        \item (I) 求线性方程组 $Ax = 0$ 的一个基础解系;
        \item (II) 求满足 $AB = E$ 的所有矩阵 $B$。
    \end{enumerate}
    
    \begin{solution}
    【详解】
    \end{solution}
\end{enumerate}

\subsection{公共解的判定与计算}

\begin{enumerate}[label=\arabic*.,start=12]
    \item (2007, 数三) 设线性方程组
    \begin{align*}
    (I) \begin{cases}
    x_1 + x_2 + x_3 = 0 \\
    x_1 + 2x_2 + a x_3 = 0 \\
    x_1 + 4x_2 + a^2 x_3 = 0
    \end{cases}
    \end{align*}
    与方程
    \begin{align*}
    (II) x_1 + 2x_2 + x_3 = a - 1
    \end{align*}
    有公共解,求 $a$ 的值及所有公共解。
    
    \begin{solution}
    【详解】
    \end{solution}
    
    \item 设齐次线性方程组
    \begin{align*}
    (I) \begin{cases}
    2x_1 + 3x_2 - x_3 = 0 \\
    x_1 + 2x_2 + x_3 - x_4 = 0
    \end{cases}
    \end{align*}
    齐次线性方程组 (II) 的一个基础解系为 $\alpha_1 = (2, -1, a+2, 1)^T$, $\alpha_2 = (-1, 2, 4, a+8)^T$ 
    \begin{enumerate}
        \item (1) 求方程组 (I) 的一个基础解系;
        \item (2) 当 $a$ 为何值时,方程组 (I) 与 (II) 有非零公共解,并求所有非零公共解。
    \end{enumerate}
    
    \begin{solution}
    【详解】
    \end{solution}
    
    \item (2005,数三) 设线性方程组
    \begin{align*}
    (I) \begin{cases}
    x_1 + 2x_2 + 3x_3 = 0 \\
    2x_1 + 3x_2 + 5x_3 = 0 \\
    x_1 + x_2 + a x_3 = 0
    \end{cases}
    \end{align*}
    与 (II) 
    \begin{align*}
    \begin{cases}
    x_1 + b x_2 + c x_3 = 0 \\
    2x_1 + b^2 x_2 + (c+1) x_3 = 0
    \end{cases}
    \end{align*}
    同解,求 $a, b, c$ 的值。
    
    \begin{solution}
    【详解】
    \end{solution}
\end{enumerate}

\section{第特征值与特征向量}

\subsection{特征值与特征向量的计算}

\begin{enumerate}[label=\arabic*.]
    \item 设
    \begin{align*}
    A = \begin{pmatrix}
    -1 & -1 & -1 \\
    -1 & -1 & -1 \\
    -1 & -1 & -1
    \end{pmatrix}
    \end{align*}
    求 $A$ 的特征值与特征向量。
    
    \begin{solution}
    【详解】
    \end{solution}
    
    \item (2003, 数一) 设
    \begin{align*}
    A = \begin{pmatrix}
    5 & -2 & 3 \\
    3 & 2 & 0 \\
    2 & 0 & 1
    \end{pmatrix}, \quad
    B = P^{-1} A^* P
    \end{align*}
    求 $B + 2E$ 的特征值与特征向量。
    
    \begin{solution}
    【详解】
    \end{solution}
    
    \item 设
    \begin{align*}
    A = \begin{pmatrix}
    1 & 2 & 2 \\
    -1 & 4 & -2 \\
    1 & -2 & a
    \end{pmatrix}
    \end{align*}
    的特征方程有一个二重根,求 $A$ 的特征值与特征向量。
    
    \begin{solution}
    【详解】
    \end{solution}
    
    \item 设 3 阶非零矩阵 $A$ 满足 $A^2 = O$,则 $A$ 的线性无关的特征向量的个数是
    \begin{enumerate}
        \item (A) 0
        \item (B) 1
        \item (C) 2
        \item (D) 3
    \end{enumerate}
    
    \begin{solution}
    【详解】
    \end{solution}
    
    \item 设 $A = \alpha \beta^T + \beta \alpha^T$,其中 $\alpha, \beta$ 为 3 维单位列向量,且 $\alpha^T \beta = \frac{1}{3}$,证明:
    \begin{enumerate}
        \item (I) 0 为 $A$ 的特征值;
        \item (II) $\alpha + \beta, \alpha - \beta$ 为 $A$ 的特征向量;
        \item (III) $A$ 可相似对角化。
    \end{enumerate}
    
    \begin{solution}
    【详解】
    \end{solution}
\end{enumerate}

\subsection{相似的判定与计算}

\begin{enumerate}[label=\arabic*.,start=6]
    \item (2019, 数一、二、三) 设
    \begin{align*}
    A = \begin{pmatrix}
    2 & 2 & 1 \\
    2 & 0 & 0 \\
    -1 & -1 & -2
    \end{pmatrix}, \quad
    B = \begin{pmatrix}
    -1 & 0 & 0 \\
    0 & -2 & 0 \\
    0 & 0 & -3
    \end{pmatrix}
    \end{align*}
    (I) 求 $x, y$ 的值;
    (II) 求可逆矩阵 $P$,使得 $P^{-1}AP = B$。
    
    \begin{solution}
    【详解】
    \end{solution}
    
    \item 设 $n$ 阶矩阵 $A$ 与 $B$ 相似,满足 $A^2 = 2E$,则 $|AB + A - B - E| = $ \underline{\hspace{3cm}}。
    
    \begin{solution}
    【详解】
    \end{solution}
\end{enumerate}

\subsection{相似对角化的判定与计算}

\begin{enumerate}[label=\arabic*.,start=8]
    \item (2005, 数一、二) 设 3 阶矩阵 $A$ 的特征值为 1, 3, -2,对应的特征向量分别为 $\alpha_1, \alpha_2, \alpha_3$。若
    \begin{align*}
    P = (\alpha_1, 2\alpha_2, -\alpha_3)
    \end{align*}
    则 $P^{-1}AP = $ \underline{\hspace{3cm}}。
    
    \begin{solution}
    【详解】
    \end{solution}
    
    \item 设 $n$ 阶方阵 $A$ 满足 $A^2 - 3A + 2E = O$,证明 $A$ 可相似对角化。
    
    \begin{solution}
    【详解】
    \end{solution}
    
    \item (2020, 数一、二、三) 设 $A$ 为 2 阶矩阵,$P = (\alpha, A\alpha)$,其中 $\alpha$ 为非零向量且不是 $A$ 的特征向量。
    \begin{enumerate}
        \item (I) 证明 $P$ 为可逆矩阵;
        \item (II) 若 $A^2\alpha + 6A\alpha - 10\alpha = 0$,求 $P^{-1}AP$,并判断 $A$ 是否相似于对角矩阵。
    \end{enumerate}
    
    \begin{solution}
    【详解】
    \end{solution}
\end{enumerate}

\subsection{实对称矩阵的计算}

\begin{enumerate}[label=\arabic*.,start=11]
    \item (2010, 数二、三) 设
    \begin{align*}
    A = \begin{pmatrix}
    0 & 1 & 4 & 1 \\
    1 & 3 & a & 1 \\
    4 & a & 0 & 1 \\
    1 & 1 & 1 & 0
    \end{pmatrix}
    \end{align*}
    正交矩阵 $Q$ 使得 $Q^T A Q$ 为对角矩阵。若 $Q$ 的第 1 列为 $\frac{1}{\sqrt{6}}(1,2,1,0)^T$,求 $a, Q$。
    
    \begin{solution}
    【详解】
    \end{solution}
    
    \item 设 3 阶实对称矩阵 $A$ 满足 $A^2 + A = O$,$A$ 的各行元素之和均为零,且 $r(A) = 2$。
    \begin{enumerate}
        \item (I) 求 $A$ 的特征值与特征向量;
        \item (II) 求矩阵 $A$。
    \end{enumerate}
    
    \begin{solution}
    【详解】
    \end{solution}
\end{enumerate}

\section{二次型}

\subsection{求二次型的标准形}

\begin{enumerate}[label=\arabic*.]
    \item (2016,数二、三) 设二次型 $ f(x_1, x_2, x_3) = a x_1^2 + a x_2^2 + (a-1) x_3^2 + 2 x_1 x_3 - 2 x_2 x_3 $ 的正、负惯性指数分别为 1,2,则
    \begin{enumerate}
        \item  $ a > 1 $
        \item  $ a < -1 $
        \item  $ -1 < a < 1 $
        \item  $ a = 1 $ 或 $ a = -1 $
    \end{enumerate}
    
    \begin{solution}
    【详解】
    \end{solution}
    
    \item (2022,数一) 设二次型 $ f(x_1, x_2, x_3) = \sum_{i=1}^3 \sum_{j=1}^3 i j x_i x_j $。
    \begin{enumerate}
        \item 求 $ f(x_1, x_2, x_3) $ 对应的矩阵;
        \item 求正交变换 $ x = Q y $,将 $ f(x_1, x_2, x_3) $ 化为标准形;
        \item 求 $ f(x_1, x_2, x_3) = 0 $ 的解。
    \end{enumerate}
    
    \begin{solution}
    【详解】
    \end{solution}
    
    \item (2020,数一、三) 设二次型 $ f(x_1, x_2) = 4 x_1^2 + 4 x_2^2 + 4 x_1 x_2 $ 经正交变换 $ \begin{pmatrix} x_1 \\ x_2 \end{pmatrix} = Q \begin{pmatrix} y_1 \\ y_2 \end{pmatrix} $ 化为二次型 $ g(y_1, y_2) = y_1^2 + b y_2^2 $,其中 $ b \geq 0 $。 
    \begin{enumerate}
        \item 求 $ a, b $ 的值;
        \item 求正交矩阵 $ Q $。
    \end{enumerate}
    
    \begin{solution}
    【详解】
    \end{solution}
\end{enumerate}

\subsection{合同的判定}

\begin{enumerate}[label=\arabic*.,start=4]
    \item (2008,数二、三) 设 $ A = \begin{pmatrix} 1 & 2 \\ 2 & 1 \end{pmatrix} $,与 $ A $ 合同的矩阵是
    \begin{enumerate}
        \item $ \begin{pmatrix} 1 & 1 \\ 1 & 2 \end{pmatrix} $
        \item $ \begin{pmatrix} 2 & 1 \\ 1 & 2 \end{pmatrix} $
        \item $ \begin{pmatrix} 2 & 1 \\ 1 & 1 \end{pmatrix} $
        \item $ \begin{pmatrix} 1 & 2 \\ 1 & 2 \end{pmatrix} $
    \end{enumerate}
    
    \begin{solution}
    【详解】
    \end{solution}
    
    \item 设 $ A, B $ 为 $ n $ 阶实对称可逆矩阵,则存在 $ n $ 阶可逆矩阵 $ P $,使得
    \begin{enumerate}
        \item $ P A P = B $;
        \item $ P^{-1} A B P = B A $;
        \item $ P^{-1} A P = B $;
        \item $ P^T A P = B $。
    \end{enumerate}
    成立的个数是
    \begin{enumerate}
        \item 1
        \item 2
        \item 3
        \item 4
    \end{enumerate}
    
    \begin{solution}
    【详解】
    \end{solution}
\end{enumerate}

\subsection{二次型正定与正定矩阵的判定}

\begin{enumerate}[label=\arabic*.,start=6]
    \item (2017,数一、二、三) 设 $ A $ 为 $ m \times n $ 阶矩阵,且 $ r(A) = n $,则下列结论
    \begin{enumerate}
        \item $ A^T A $ 与单位矩阵等价;
        \item $ A^T A $ 与对角矩阵相似;
        \item $ A^T A $ 与单位矩阵合同;
        \item $ A^T A $ 正定。
    \end{enumerate}
    正确的个数是
    \begin{enumerate}
        \item 1
        \item 2
        \item 3
        \item 4
    \end{enumerate}
    
    \begin{solution}
    【详解】
    \end{solution}
    
    \item 证明:
    \begin{enumerate}
        \item 设 $ A $ 为 $ n $ 阶正定矩阵,$ B $ 为 $ n $ 阶反对称矩阵,则 $ A - B^2 $ 为正定矩阵;
        \item 设 $ A, B $ 为 $ n $ 阶矩阵,且 $ r(A + B) = n $,则 $ A^T A + B^T B $ 为正定矩阵。
    \end{enumerate}
    
    \begin{solution}
    【详解】
    \end{solution}
\end{enumerate}

\ifx\allfiles\undefined
\end{document}
\fi