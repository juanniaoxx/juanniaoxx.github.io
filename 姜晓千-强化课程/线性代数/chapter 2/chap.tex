\ifx\allfiles\undefined
\documentclass[12pt, a4paper, oneside, UTF8]{ctexbook}
\usepackage{multirow}
\def\path{../../config}
\input{../../config/_config}
\begin{document}
% \input{../config/cover}
\else
\fi

\chapter{矩阵}
\section{求高次幂}

\begin{enumerate}[label=\arabic*.]
    \item 设 $ A = \sqrt{a} $,$ B $ 为3阶矩阵,满足 $ BA = O $,且 $ r(B) > 1 $,则 $ A^n = 0 $。
    
    \begin{solution}
    【详解】
    \end{solution}
    
    \item 设 
    \begin{align*}
    A = \begin{pmatrix}
    2 & 0 & 0 \\
    0 & -3 & 2 \\
    0 & 4 & 1
    \end{pmatrix}
    \end{align*}
    则 $ A^n = $ \underline{\hspace{3cm}}。
    
    \begin{solution}
    【详解】
    \end{solution}
    
    \item 设 
    \begin{align*}
    A = \begin{pmatrix}
    -1 & 2 & -1 \\
    -1 & 2 & -1 \\
    -3 & 6 & -3
    \end{pmatrix}
    \end{align*}
    $ P $ 为3阶可逆矩阵,$ B = P^{-1}AP $,则 $ (B + E)^{100} = $ \underline{\hspace{3cm}}。
    
    \begin{solution}
    【详解】
    \end{solution}
\end{enumerate}

\section{逆的判定与计算}

\begin{enumerate}[label=\arabic*.,start=3]
    \item 设 $ n $ 阶矩阵 $ A $ 满足 $ A^2 = 2A $,则下列结论不正确的是:
    
    \begin{solution}
    【详解】
    \end{solution}
    
    \item 设 $ A, B $ 为 $ n $ 阶矩阵,$ a, b $ 为非零常数。证明:
    \begin{enumerate}
        \item 若 $ AB = aA + bB $,则 $ AB = BA $;
        \item 若 $ A^2 + aAB = E $,则 $ AB = BA $。
    \end{enumerate}
    
    \begin{solution}
    【详解】
    \end{solution}
    
    \item 设 
    \begin{align*}
    A = \begin{pmatrix}
    a & 1 & 0 \\
    1 & a & -1 \\
    0 & 1 & a
    \end{pmatrix}
    \end{align*}
    满足 $ A^3 = O $。
    \begin{enumerate}
        \item 求 $ a $ 的值;
        \item 若矩阵 $ X $ 满足 $ X - XA^2 - AX + AXA^2 = E $,求 $ X $。
    \end{enumerate}
    
    \begin{solution}
    【详解】
    \end{solution}
\end{enumerate}

\section{秩的计算与证明}

\begin{enumerate}[label=\arabic*.,start=6]
    \item (2018, 数一、二、三) 设 $ A, B $ 为 $ n $ 阶矩阵,$ (XY) $ 表示分块矩阵,则:
    \begin{enumerate}
        \item $ r(A A B) = r(A) $
        \item $ r(A B A) = r(A) $
        \item $ r(A B) = \max\{r(A), r(B)\} $
        \item $ r(A B) = r(A^T B^T) $
    \end{enumerate}
    
    \begin{solution}
    【详解】
    \end{solution}
    
    \item (1) 若 $ A^2 = A $,则 $ r(A) + r(A - E) = n $。
    \item (II) 若 $ A^2 = E $,则 $ r(A + E) + r(A - E) = n $。
    
    \begin{solution}
    【详解】
    \end{solution}
\end{enumerate}

\section{关于伴随矩阵}

\begin{enumerate}[label=\arabic*.,start=8]
    \item 设 $ n $ 阶矩阵 $ A $ 的各列元素之和均为 2,且 $ |A| = 6 $,则 $ A^* $ 的各列元素之和均为:
    \begin{enumerate}
        \item (A) 2
        \item (B) 1
        \item (C) 3
        \item (D) 6
    \end{enumerate}
    
    \begin{solution}
    【详解】
    \end{solution}
    
    \item 设 $ A = (a_{ij}) $ 为 $ n(n \geq 3) $ 阶非零矩阵,$ A_{ij} $ 为 $ a_{ij} $ 的代数余子式,证明:
    \begin{enumerate}
        \item $ a_{ij} = A_{ij}(i, j = 1, 2, \cdots, n) \Leftrightarrow A^* = A^T \Leftrightarrow AA^T = E $ 且 $ |A| = 1 $;
        \item $ a_{ij} = -A_{ij}(i, j = 1, 2, \cdots, n) \Leftrightarrow A^* = -A^T \Leftrightarrow AA^T = E $ 且 $ |A| = -1 $。
    \end{enumerate}
    
    \begin{solution}
    【详解】
    \end{solution}
\end{enumerate}

\section{初等变换与初等矩阵}

\begin{enumerate}[label=\arabic*.,start=10]
    \item (2005, 数一、二) 设 $ A $ 为 $ n(n \geq 2) $ 阶可逆矩阵,交换 $ A $ 的第 1 行与第 2 行得到矩阵 $ B $,则:
    \begin{enumerate}
        \item (A) 交换 $ A^* $ 的第 1 列与第 2 列,得 $ B^* $
        \item (C) 交换 $ A^* $ 的第 1 列与第 2 列,得 $ -B^* $
        \item (D) 交换 $ A $ 的第 1 行与第 2 行,得 $ -B^* $
    \end{enumerate}
    
    \begin{solution}
    【详解】
    \end{solution}
    
    \item 设 
    \begin{align*}
    A = \begin{pmatrix}
    2 & 3 & 0 & 0 \\
    1 & 2 & 0 & 0 \\
    0 & 0 & 0 & 0 \\
    0 & 0 & 0 & 0
    \end{pmatrix}, \quad
    P = \begin{pmatrix}
    0 & 1 & 0 & 0 \\
    1 & 0 & 0 & 0 \\
    0 & 0 & 1 & 0 \\
    0 & 0 & 0 & 1
    \end{pmatrix}, \quad
    Q = \begin{pmatrix}
    1 & 1 & 0 \\
    0 & 1 & 0 \\
    1 & 0 & 0
    \end{pmatrix}
    \end{align*}
    则 $ (P^{-1})^{2023} A (Q^T)^{2022} = $ \underline{\hspace{3cm}}。
    
    \begin{solution}
    【详解】
    \end{solution}
\end{enumerate}

\ifx\allfiles\undefined
\end{document}
\fi