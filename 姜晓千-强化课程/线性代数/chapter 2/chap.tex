\ifx\allfiles\undefined
\documentclass[12pt, a4paper, oneside, UTF8]{ctexbook}
\usepackage{multirow}
\def\path{../../config}
\usepackage{amsmath}
\usepackage{amsthm}
\usepackage{amssymb}
\usepackage{array}
\usepackage{xcolor}
\usepackage{graphicx}
\usepackage{mathrsfs}
\usepackage{enumitem}
\usepackage{geometry}
\usepackage[colorlinks, linkcolor=black]{hyperref}
\usepackage{stackengine}
\usepackage{yhmath}
\usepackage{extarrows}
\usepackage{tikz}
\usepackage{pgfplots}
\usetikzlibrary{decorations.pathreplacing, positioning}
\usepgfplotslibrary{fillbetween}
% \usepackage{unicode-math}
\usepackage{esint}
\usepackage[most]{tcolorbox}

\usepackage{fancyhdr}
\usepackage[dvipsnames, svgnames]{xcolor}
\usepackage{listings}

\definecolor{mygreen}{rgb}{0,0.6,0}
\definecolor{mygray}{rgb}{0.5,0.5,0.5}
\definecolor{mymauve}{rgb}{0.58,0,0.82}
\definecolor{NavyBlue}{RGB}{0,0,128}
\definecolor{Rhodamine}{RGB}{255,0,255}
\definecolor{PineGreen}{RGB}{0,128,0}

\graphicspath{ {figures/},{../figures/}, {config/}, {../config/} }

\linespread{1.6}

\geometry{
    top=25.4mm, 
    bottom=25.4mm, 
    left=20mm, 
    right=20mm, 
    headheight=2.17cm, 
    headsep=4mm, 
    footskip=12mm
}

\setenumerate[1]{itemsep=5pt,partopsep=0pt,parsep=\parskip,topsep=5pt}
\setitemize[1]{itemsep=5pt,partopsep=0pt,parsep=\parskip,topsep=5pt}
\setdescription{itemsep=5pt,partopsep=0pt,parsep=\parskip,topsep=5pt}

\lstset{
    language=Mathematica,
    basicstyle=\tt,
    breaklines=true,
    keywordstyle=\bfseries\color{NavyBlue}, 
    emphstyle=\bfseries\color{Rhodamine},
    commentstyle=\itshape\color{black!50!white}, 
    stringstyle=\bfseries\color{PineGreen!90!black},
    columns=flexible,
    numbers=left,
    numberstyle=\footnotesize,
    frame=tb,
    breakatwhitespace=false,
} 

\lstset{
    language=TeX, % 设置语言为 TeX
    basicstyle=\ttfamily, % 使用等宽字体
    breaklines=true, % 自动换行
    keywordstyle=\bfseries\color{NavyBlue}, % 关键字样式
    emphstyle=\bfseries\color{Rhodamine}, % 强调样式
    commentstyle=\itshape\color{black!50!white}, % 注释样式
    stringstyle=\bfseries\color{PineGreen!90!black}, % 字符串样式
    columns=flexible, % 列的灵活性
    numbers=left, % 行号在左侧
    numberstyle=\footnotesize, % 行号字体大小
    frame=tb, % 顶部和底部边框
    breakatwhitespace=false % 不在空白处断行
}

% \begin{lstlisting}[language=TeX] ... \end{lstlisting}

% 定理环境设置
\usepackage[strict]{changepage} 
\usepackage{framed}

\definecolor{greenshade}{rgb}{0.90,1,0.92}
\definecolor{redshade}{rgb}{1.00,0.88,0.88}
\definecolor{brownshade}{rgb}{0.99,0.95,0.9}
\definecolor{lilacshade}{rgb}{0.95,0.93,0.98}
\definecolor{orangeshade}{rgb}{1.00,0.88,0.82}
\definecolor{lightblueshade}{rgb}{0.8,0.92,1}
\definecolor{purple}{rgb}{0.81,0.85,1}

\theoremstyle{definition}
\newtheorem{myDefn}{\indent Definition}[section]
\newtheorem{myLemma}{\indent Lemma}[section]
\newtheorem{myThm}[myLemma]{\indent Theorem}
\newtheorem{myCorollary}[myLemma]{\indent Corollary}
\newtheorem{myCriterion}[myLemma]{\indent Criterion}
\newtheorem*{myRemark}{\indent Remark}
\newtheorem{myProposition}{\indent Proposition}[section]

\newenvironment{formal}[2][]{%
	\def\FrameCommand{%
		\hspace{1pt}%
		{\color{#1}\vrule width 2pt}%
		{\color{#2}\vrule width 4pt}%
		\colorbox{#2}%
	}%
	\MakeFramed{\advance\hsize-\width\FrameRestore}%
	\noindent\hspace{-4.55pt}%
	\begin{adjustwidth}{}{7pt}\vspace{2pt}\vspace{2pt}}{%
		\vspace{2pt}\end{adjustwidth}\endMakeFramed%
}

\newenvironment{definition}{\vspace{-\baselineskip * 2 / 3}%
	\begin{formal}[Green]{greenshade}\vspace{-\baselineskip * 4 / 5}\begin{myDefn}}
	{\end{myDefn}\end{formal}\vspace{-\baselineskip * 2 / 3}}

\newenvironment{theorem}{\vspace{-\baselineskip * 2 / 3}%
	\begin{formal}[LightSkyBlue]{lightblueshade}\vspace{-\baselineskip * 4 / 5}\begin{myThm}}%
	{\end{myThm}\end{formal}\vspace{-\baselineskip * 2 / 3}}

\newenvironment{lemma}{\vspace{-\baselineskip * 2 / 3}%
	\begin{formal}[Plum]{lilacshade}\vspace{-\baselineskip * 4 / 5}\begin{myLemma}}%
	{\end{myLemma}\end{formal}\vspace{-\baselineskip * 2 / 3}}

\newenvironment{corollary}{\vspace{-\baselineskip * 2 / 3}%
	\begin{formal}[BurlyWood]{brownshade}\vspace{-\baselineskip * 4 / 5}\begin{myCorollary}}%
	{\end{myCorollary}\end{formal}\vspace{-\baselineskip * 2 / 3}}

\newenvironment{criterion}{\vspace{-\baselineskip * 2 / 3}%
	\begin{formal}[DarkOrange]{orangeshade}\vspace{-\baselineskip * 4 / 5}\begin{myCriterion}}%
	{\end{myCriterion}\end{formal}\vspace{-\baselineskip * 2 / 3}}
	

\newenvironment{remark}{\vspace{-\baselineskip * 2 / 3}%
	\begin{formal}[LightCoral]{redshade}\vspace{-\baselineskip * 4 / 5}\begin{myRemark}}%
	{\end{myRemark}\end{formal}\vspace{-\baselineskip * 2 / 3}}

\newenvironment{proposition}{\vspace{-\baselineskip * 2 / 3}%
	\begin{formal}[RoyalPurple]{purple}\vspace{-\baselineskip * 4 / 5}\begin{myProposition}}%
	{\end{myProposition}\end{formal}\vspace{-\baselineskip * 2 / 3}}


\newtheorem{example}{\indent \color{SeaGreen}{Example}}[section]
\renewcommand{\proofname}{\indent\textbf{\textcolor{TealBlue}{Proof}}}
\newenvironment{solution}{\begin{proof}[\indent\textbf{\textcolor{TealBlue}{Solution}}]}{\end{proof}}

% 自定义命令的文件

\def\d{\mathrm{d}}
\def\R{\mathbb{R}}
%\newcommand{\bs}[1]{\boldsymbol{#1}}
%\newcommand{\ora}[1]{\overrightarrow{#1}}
\newcommand{\myspace}[1]{\par\vspace{#1\baselineskip}}
\newcommand{\xrowht}[2][0]{\addstackgap[.5\dimexpr#2\relax]{\vphantom{#1}}}
\newenvironment{mycases}[1][1]{\linespread{#1} \selectfont \begin{cases}}{\end{cases}}
\newenvironment{myvmatrix}[1][1]{\linespread{#1} \selectfont \begin{vmatrix}}{\end{vmatrix}}
\newcommand{\tabincell}[2]{\begin{tabular}{@{}#1@{}}#2\end{tabular}}
\newcommand{\pll}{\kern 0.56em/\kern -0.8em /\kern 0.56em}
\newcommand{\dive}[1][F]{\mathrm{div}\;\boldsymbol{#1}}
\newcommand{\rotn}[1][A]{\mathrm{rot}\;\boldsymbol{#1}}

% 修改参数改变封面样式,0 默认原始封面、内置其他1、2、3种封面样式
\def\myIndex{0}


\ifnum\myIndex>0
    \input{\path/cover_package_\myIndex} 
\fi

\def\myTitle{姜晓千 2023年强化班笔记}
\def\myAuthor{Weary Bird}
\def\myDateCover{\today}
\def\myDateForeword{\today}
\def\myForeword{相见欢·林花谢了春红}
\def\myForewordText{
    林花谢了春红,太匆匆。
    无奈朝来寒雨晚来风。
    胭脂泪,相留醉,几时重。
    自是人生长恨水长东。
}
\def\mySubheading{数学笔记}


\begin{document}
% \input{\path/cover_text_\myIndex.tex}

\newpage
\thispagestyle{empty}
\begin{center}
    \Huge\textbf{\myForeword}
\end{center}
\myForewordText
\begin{flushright}
    \begin{tabular}{c}
        \myDateForeword
    \end{tabular}
\end{flushright}

\newpage
\pagestyle{plain}
\setcounter{page}{1}
\pagenumbering{Roman}
\tableofcontents

\newpage
\pagenumbering{arabic}
% \setcounter{chapter}{-1}
\setcounter{page}{1}

\pagestyle{fancy}
\fancyfoot[C]{\thepage}
\renewcommand{\headrulewidth}{0.4pt}
\renewcommand{\footrulewidth}{0pt}








\else
\fi

\chapter{矩阵}
\section{求高次幂}

\begin{enumerate}[label=\arabic*.]
    \item 设 $ A = \sqrt{a} $,$ B $ 为3阶矩阵,满足 $ BA = O $,且 $ r(B) > 1 $,则 $ A^n = 0 $。
    
    \begin{solution}
    【详解】
    \end{solution}
    
    \item 设 
    \begin{align*}
    A = \begin{pmatrix}
    2 & 0 & 0 \\
    0 & -3 & 2 \\
    0 & 4 & 1
    \end{pmatrix}
    \end{align*}
    则 $ A^n = $ \underline{\hspace{3cm}}。
    
    \begin{solution}
    【详解】
    \end{solution}
    
    \item 设 
    \begin{align*}
    A = \begin{pmatrix}
    -1 & 2 & -1 \\
    -1 & 2 & -1 \\
    -3 & 6 & -3
    \end{pmatrix}
    \end{align*}
    $ P $ 为3阶可逆矩阵,$ B = P^{-1}AP $,则 $ (B + E)^{100} = $ \underline{\hspace{3cm}}。
    
    \begin{solution}
    【详解】
    \end{solution}
\end{enumerate}

\section{逆的判定与计算}

\begin{enumerate}[label=\arabic*.,start=3]
    \item 设 $ n $ 阶矩阵 $ A $ 满足 $ A^2 = 2A $,则下列结论不正确的是:
    
    \begin{solution}
    【详解】
    \end{solution}
    
    \item 设 $ A, B $ 为 $ n $ 阶矩阵,$ a, b $ 为非零常数。证明:
    \begin{enumerate}
        \item 若 $ AB = aA + bB $,则 $ AB = BA $;
        \item 若 $ A^2 + aAB = E $,则 $ AB = BA $。
    \end{enumerate}
    
    \begin{solution}
    【详解】
    \end{solution}
    
    \item 设 
    \begin{align*}
    A = \begin{pmatrix}
    a & 1 & 0 \\
    1 & a & -1 \\
    0 & 1 & a
    \end{pmatrix}
    \end{align*}
    满足 $ A^3 = O $。
    \begin{enumerate}
        \item 求 $ a $ 的值;
        \item 若矩阵 $ X $ 满足 $ X - XA^2 - AX + AXA^2 = E $,求 $ X $。
    \end{enumerate}
    
    \begin{solution}
    【详解】
    \end{solution}
\end{enumerate}

\section{秩的计算与证明}

\begin{enumerate}[label=\arabic*.,start=6]
    \item (2018, 数一、二、三) 设 $ A, B $ 为 $ n $ 阶矩阵,$ (XY) $ 表示分块矩阵,则:
    \begin{enumerate}
        \item $ r(A A B) = r(A) $
        \item $ r(A B A) = r(A) $
        \item $ r(A B) = \max\{r(A), r(B)\} $
        \item $ r(A B) = r(A^T B^T) $
    \end{enumerate}
    
    \begin{solution}
    【详解】
    \end{solution}
    
    \item (1) 若 $ A^2 = A $,则 $ r(A) + r(A - E) = n $。
    \item (II) 若 $ A^2 = E $,则 $ r(A + E) + r(A - E) = n $。
    
    \begin{solution}
    【详解】
    \end{solution}
\end{enumerate}

\section{关于伴随矩阵}

\begin{enumerate}[label=\arabic*.,start=8]
    \item 设 $ n $ 阶矩阵 $ A $ 的各列元素之和均为 2,且 $ |A| = 6 $,则 $ A^* $ 的各列元素之和均为:
    \begin{enumerate}
        \item (A) 2
        \item (B) 1
        \item (C) 3
        \item (D) 6
    \end{enumerate}
    
    \begin{solution}
    【详解】
    \end{solution}
    
    \item 设 $ A = (a_{ij}) $ 为 $ n(n \geq 3) $ 阶非零矩阵,$ A_{ij} $ 为 $ a_{ij} $ 的代数余子式,证明:
    \begin{enumerate}
        \item $ a_{ij} = A_{ij}(i, j = 1, 2, \cdots, n) \Leftrightarrow A^* = A^T \Leftrightarrow AA^T = E $ 且 $ |A| = 1 $;
        \item $ a_{ij} = -A_{ij}(i, j = 1, 2, \cdots, n) \Leftrightarrow A^* = -A^T \Leftrightarrow AA^T = E $ 且 $ |A| = -1 $。
    \end{enumerate}
    
    \begin{solution}
    【详解】
    \end{solution}
\end{enumerate}

\section{初等变换与初等矩阵}

\begin{enumerate}[label=\arabic*.,start=10]
    \item (2005, 数一、二) 设 $ A $ 为 $ n(n \geq 2) $ 阶可逆矩阵,交换 $ A $ 的第 1 行与第 2 行得到矩阵 $ B $,则:
    \begin{enumerate}
        \item (A) 交换 $ A^* $ 的第 1 列与第 2 列,得 $ B^* $
        \item (C) 交换 $ A^* $ 的第 1 列与第 2 列,得 $ -B^* $
        \item (D) 交换 $ A $ 的第 1 行与第 2 行,得 $ -B^* $
    \end{enumerate}
    
    \begin{solution}
    【详解】
    \end{solution}
    
    \item 设 
    \begin{align*}
    A = \begin{pmatrix}
    2 & 3 & 0 & 0 \\
    1 & 2 & 0 & 0 \\
    0 & 0 & 0 & 0 \\
    0 & 0 & 0 & 0
    \end{pmatrix}, \quad
    P = \begin{pmatrix}
    0 & 1 & 0 & 0 \\
    1 & 0 & 0 & 0 \\
    0 & 0 & 1 & 0 \\
    0 & 0 & 0 & 1
    \end{pmatrix}, \quad
    Q = \begin{pmatrix}
    1 & 1 & 0 \\
    0 & 1 & 0 \\
    1 & 0 & 0
    \end{pmatrix}
    \end{align*}
    则 $ (P^{-1})^{2023} A (Q^T)^{2022} = $ \underline{\hspace{3cm}}。
    
    \begin{solution}
    【详解】
    \end{solution}
\end{enumerate}

\ifx\allfiles\undefined
\end{document}
\fi